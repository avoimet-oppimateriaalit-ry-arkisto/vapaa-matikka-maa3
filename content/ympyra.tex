\section*{Ympyrä}

\begin{luoKuva}{ympyraosat}
	skaalaa(2)
	keskipiste = geom.piste(0, 0, "keskipiste")
	ympyra = geom.ympyra(keskipiste, 1, r"ympyr\"{a}", 30)
	kehapiste = geom.leikkauspiste(
		ympyra, geom.suoraSuuntaan(keskipiste, -1, 2, piirra = False), piirra = False
	)
	geom.jana(kehapiste, keskipiste, r"s\"{a}de")
\end{luoKuva}
\laatikko{
\termi{ympyrä}{Ympyrä} on sellaisten pisteiden joukko, jotka ovat kiinteällä etäisyydellä tietystä pisteestä. Tätä kiinteää etäisyyttä kutsutaan ympyrän \termi{säde}{säteeksi} ja kyseistä pistettä ympyrän \termi{keskipiste}{keskipisteeksi}.
}
\begin{center}
\naytaKuva{ympyraosat}
\end{center}

\begin{esimerkki} %FIXME Mikä tämän esimerkin idea on? Voisi varmaankin jättää pois.
Kuu kiertää Maata likimäärin ympyränmuotoisella radalla, jonka keskipiste on Maa ja säde on Kuun ja Maan välinen etäisyys.
\end{esimerkki}

(KUVA TÄHÄN)

\begin{luoKuva}{ympyrajanne}
	skaalaa(2)
	O = geom.piste(0, 0)
	w = geom.ympyra(O, 1)
	s = geom.suoraSuuntaan(O, 1, 0, piirra = False)
	l = geom.suoraSuuntaan((0.2, 0.6), 1, -0.2, piirra = False)
	A = geom.leikkauspiste(w, s, valinta = False, piirra = False)
	B = geom.leikkauspiste(w, s, valinta = True, piirra = False)
	C = geom.leikkauspiste(w, l, valinta = False, piirra = False)
	D = geom.leikkauspiste(w, l, valinta = True, piirra = False)
	geom.jana(A, B, "halkaisija")
	geom.jana(C, D, r"j\"{a}nne")
\end{luoKuva}
\laatikko{
Ympyrän \termi{jänne}{jänne} on jana kahden ympyrän kehäpisteen välillä. Mikäli jänne kulkee ympyrän keskipisteen kautta, se on ympyrän \termi{halkaisija}{halkaisija}. Halkaisijan pituus on kaksi kertaa ympyrän säde.
}
%toinen tapa määritellä olisi sekantin avulla


\begin{center}
\naytaKuva{ympyrajanne}
\end{center}

Ympyrät ovat keskenään yhdenmuotoisia, joten ympyrän sädettä kasvattaessa ympyrän halkaisija kasvaa samassa suhteessa. Siis ympyrän kehän pituuden suhde ympyrän säteeseen on vakio. Tätä vakiota kutsutaan piiksi ja sitä merkitään kirjaimella $\pi$, ja sen arvo on likimain $3,14$. Piin määritelmän nojalla saadaan luonnollisesti kaava ympyrän kehän pituudelle:

% FIXME: onko $\pi$ rationaaliluku, piin määritelmä erikseen ja alla olevaan laatikkoon vain p=2\pi r

\laatikko{
Jos ympyrän halkaisija on $d$, sen kehän pituus $p$ saadaan kaavasta
\[p = \pi d.\]
Siis jos ympyrän säde on $r$, myös
\[p = 2 \pi r.\]
}

% FIXME tämä turhaa?
%Tasokuvioita kasvattaessa niiden pinta-alat kasvavat kertoimen toisessa potenssissa, joten myös ympyrän pinta-alan ja säteen neliön suhde pysyvät vakiona. %Käy ilmi että tämäkin suhde on $\pi$.

\laatikko{
% FIXME: ympyrän pinta-ala = 0, kiekon pinta-ala?
Jos ympyrän säde on $r$, sen pinta-ala $A$ saadaan kaavasta
\[A = \pi r^2.\]
}

\begin{tehtavasivu}

\paragraph*{Opi perusteet}

\begin{tehtava}
Laske piiri kun säde on:
\begin{alakohdat}
\alakohta{r = 7}
\alakohta{r = 24}
\end{alakohdat}
\begin{vastaus}
\begin{alakohdat}
\alakohta{$p = 14\pi \approx 44,0$}
\alakohta{$p = 48\pi \approx 150,8$}
\end{alakohdat}
\end{vastaus}
\end{tehtava}

\begin{tehtava}
Laske ympyrän pinta-ala kun ympyrän säde on:
\begin{alakohdat}
\alakohta{$r = 5$.}
\alakohta{$r = 2$.}
\alakohta{$r = \frac35$.}
\end{alakohdat}

\begin{vastaus}
\begin{alakohdat}
\alakohta{$A = 25\pi \approx 78,5$
}
\alakohta{$A = 4\pi \approx 12,6$
}
\alakohta{$A = \frac{9}{25} \pi \approx 1,13$
}
\end{alakohdat}
\end{vastaus}
\end{tehtava}


\paragraph*{Hallitse kokonaisuus}

\paragraph*{Sekalaisia tehtäviä}

\end{tehtavasivu}
