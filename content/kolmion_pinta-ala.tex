\section*{Pinta-alan laskeminen}

Kolmion pinta-alan laskemiseen on olemassa useita tapoja. Tutuin lienee kaavan ''kanta kertaa korkeus jaettuan kahdella'' käyttö. Usein tulee kuitenkin vastaan tilanteita, joissa kolmion korkeutta ei tiedetä, mutta tiedetään esimerkiksi sivujen pituuksia ja ehkä lisäksi myös kulmien suuruuksia. Tällaisista tapauksista voi selvitä helpommalla, jos tuntee muitakin tapoja pinta-alan määrittämiseksi.

% FIXME: perustelu suunnikkail?
\begin{luoKuva}{kolmiopintaala}
	rajaa(minX = -1, maxX = 6)
	A = (0, 0)
	B = (5, 0)
	C = (3, 3)
	P = (3, 0)
	geom.jana(A, B, "$s$")
	geom.jana(B, C)
	geom.jana(C, A)
	#geom.jana(P, C, "$h$") # testi
	vari("black!10!white")
	geom.jana(P, C) 
	geom.suorakulma(C, P, A)
\end{luoKuva}
%#h jää näkyviin
% # laitetaan suora harmaalla päälle
%	vari("black!10!white")

\laatikko{
Kolmion pinta-ala on puolet kolmion sivun pituuden ja saman sivun määräämälle suoralle piirretyn korkeusjanan pituuden tulosta.
\[A = \frac12sh\]
\begin{center}
\naytaKuva{kolmiopintaala}
\end{center}
}

\begin{center}
\begin{kuva}
	rajaa(minX = -1, maxX = 6)
	A = (0, 0)
	B = (5, 0)
	C = (6, 3)
	P = (6, 0)
	geom.jana(A, B, "$s$")
	geom.jana(B, C)
	geom.jana(C, A)
	geom.jana(P, C, "$h$")
	vari("black!10!white")
	geom.jana(P, C)
	geom.suorakulma(C, P, A)
	geom.jana(B, P)
\end{kuva}
\end{center}
%FIXME kuvaan korkeusjana harmaalla/katkoviivalla, jotta kolmio erottuu paremmin


%FIXME \todo{Olisiko parempi esitellä trigonometrinen kaava vasta trigonometrian jälkeen, niin voisi perustella.}

Kolmion pinta-alan trigonometrinen kaava:

\laatikko{
Kolmion sivut ovat a,b ja c ja niitä vastaavat kulmat ovat $\alpha$, $\beta$ ja $\gamma$. Kolmion pinta-ala on
$$A= \frac{1}{2} a b \sin \gamma.$$
}

Tosielämässä on usein helpompaa mitata kolmioiden sivujen pituudet kuin kulmien suuruudet. Tällöin pinta-alan määrittämiseen voi käyttää Heronin kaavaa.

\laatikko{
Kolmion sivut ovat $a$,$b$ ja $c$. $p$ on puolet kolmion piiristä (ns. \termi{puolipiiri}{puolipiiri} eli 
\[p = \frac{a+b+c}{2} \]
Kolmion pinta-ala on 
$$A = \sqrt{p(p-a)(p-b)(p-c)}.$$
}

\begin{tehtavasivu}
\paragraph*{Opi perusteet}

\paragraph*{Hallitse kokonaisuus}

\paragraph*{Sekalaisia tehtäviä}

\begin{tehtava}
Piste $P$ on kolmion $ABC$ sisällä. Suora $AP$ leikkaa sivun $BC$ pisteessä $D$. Todista, että
\[
\frac{BD}{CD} = \frac{|BDA|}{|CDA|} = \frac{|BDP|}{|CDP|} = \frac{|BPA|}{|CPA|}
\]
missä $|BDA|$ on kolmion $BDA$ ala jne.. Todista edelleen \termi{Cevan lause}{Cevan lause}: Jos pisteet $D$, $E$ ja $F$ ovat kolmion sivuilla $BC$, $CA$ ja $AB$ vastaavasti, suorat $AD$, $BE$ ja $CF$ kulkevat saman pisteen kautta jos ja vain jos
\[
\frac{BD\cdot CE \cdot AE}{CD \cdot EA \cdot FB}
\]
\begin{vastaus}
Vinkki: Kolmiolla $BDA$ ja $CDA$ on sama korkeus, vastaavasti kolmioilla $BDP$ ja $CDP$. Tutki verrantoja. $\Rightarrow$ Cevan lauseen suhteet voidaan esittää kolmoiden $BPA$, $APC$ ja $CPB$ alojen avulla. $\Leftarrow$ Tutki pistettä, $F'$ (janalla $AB$), jolla suorat kulkevat saman pisteen kautta ja käytä jo todistettua puolta todistaaksesi, että $F' = F$.
\end{vastaus}
\end{tehtava}
\end{tehtavasivu}

