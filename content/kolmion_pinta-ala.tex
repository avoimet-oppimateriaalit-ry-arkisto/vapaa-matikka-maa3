\section*{Pinta-alan laskeminen}

Kolmion pinta-alan laskemiseen on olemassa useita tapoja. Tutuin lienee kaava ''kanta kertaa korkeus jaettuna kahdella.'' Usein tulee kuitenkin vastaan tilanteita, joissa kolmion korkeutta ei tiedetä, mutta tiedetään esimerkiksi sivujen pituuksia ja ehkä lisäksi myös kulmien suuruuksia. Tällaisista tapauksista voi selvitä helpommalla, jos tuntee muitakin tapoja pinta-alan määrittämiseksi.

% FIXME: perustelu suunnikkail?
\begin{luoKuva}{kolmiopintaala}
	rajaa(minX = -1, maxX = 6)
	A = (0, 0)
	B = (5, 0)
	C = (3, 3)
	P = (3, 0)
	geom.jana(A, B, "$s$")
	geom.jana(B, C)
	geom.jana(C, A)
	#geom.jana(P, C, "$h$") # testi
	vari("black!10!white")
	geom.jana(P, C) 
	geom.suorakulma(C, P, A)
\end{luoKuva}
%#h jää näkyviin
% # laitetaan suora harmaalla päälle
%	vari("black!10!white")

\laatikko{
Kolmion pinta-ala on puolet kolmion sivun pituuden ja saman sivun määräämälle suoralle piirretyn korkeusjanan pituuden tulosta.
\[A = \frac12sh\]
\begin{center}
\naytaKuva{kolmiopintaala}
\end{center}
}

\begin{center}
\begin{kuva}
	rajaa(minX = -1, maxX = 6)
	A = (0, 0)
	B = (5, 0)
	C = (6, 3)
	P = (6, 0)
	geom.jana(A, B, "$s$")
	geom.jana(B, C)
	geom.jana(C, A)
	geom.jana(P, C, "$h$")
	vari("black!10!white")
	geom.jana(P, C)
	geom.suorakulma(C, P, A)
	geom.jana(B, P)
\end{kuva}
\end{center}
%FIXME kuvaan korkeusjana harmaalla/katkoviivalla, jotta kolmio erottuu paremmin


%FIXME \todo{Olisiko parempi esitellä trigonometrinen kaava vasta trigonometrian jälkeen, niin voisi perustella.}

Kolmion pinta-alan trigonometrinen kaava:

\laatikko{
Kolmion sivut ovat a, b ja c ja niitä vastaavat kulmat ovat $\alpha$, $\beta$ ja $\gamma$. Kolmion pinta-ala on
$$A= \frac{1}{2} a b \sin \gamma = \frac{1}{2} b c \sin \alpha = \frac{1}{2} c a \sin \beta.$$
}

Tosielämässä on usein helpompaa mitata kolmioiden sivujen pituudet kuin kulmien suuruudet. Tällöin pinta-alan määrittämiseen voi käyttää \termi{Heronin kaavaa}{Heronin kaava}.

\laatikko{
Kolmion sivut ovat $a$, $b$ ja $c$. $p$ on puolet kolmion piiristä (ns. \termi{puolipiiri}{puolipiiri}) eli 
\[p = \frac{a+b+c}{2}. \]
Kolmion pinta-ala on 
$$A = \sqrt{p(p-a)(p-b)(p-c)}.$$
}

\begin{tehtavasivu}

\paragraph*{Opi perusteet}

\begin{tehtava}
Laske kolmion ala, kun sen kanta ja korkeus ovat
\begin{alakohdat}
	\alakohta{$ 7,6$}
	\alakohta{$ \frac{6}{5}, \frac{4}{3}$}
	\alakohta{$ \sqrt{13}, \sqrt{13}$,}
\end{alakohdat}
vastaavasti
\begin{vastaus}
\begin{alakohdat}
	\alakohta{$21$}
	\alakohta{$\frac{4}{5} = 0,8$}
	\alakohta{$\frac{13}{2} = 6,5$}
\end{alakohdat}
\end{vastaus}
\end{tehtava}

\begin{tehtava}
Kolmion $ABC$ ala on kaksi kertaa niin suuri kuin kolmion $DEF$, mutta sen korkeus on vain kolmasosa kolmion $DEF$ korkeudesta. Kuinka moninkertainen on sen on $ABC$:n kannan pituus verrattuna $DEF$ kantaan?
\begin{vastaus}
$ABC$:n kanta on kuusi kertaa niin pitkä kuin $DEF$:n kanta.
\end{vastaus}
\end{tehtava}


\begin{tehtava}
Laske kolmion ala, kun sen kaksi sivua, ja niiden välinen kulma on
\begin{alakohdat}
	\alakohta{$ 3,4, 90^{\circ} $}
	\alakohta{$ \sqrt{5}, 27, 60^{\circ} $}
	\alakohta{$ \pi , 3, 45^{\circ}. $}
\end{alakohdat}
\begin{vastaus}
\begin{alakohdat}
	\alakohta{$6$}
	\alakohta{$\frac{27\sqrt{15}}{4} \approx 26,1$}
	\alakohta{$\frac{3\pi}{2\sqrt{2}} \approx 3,33$}
\end{alakohdat}
\end{vastaus}
\end{tehtava}

\begin{tehtava}
Laske kolmion ala, kun sen sivujen pituudet ovat
\begin{alakohdat}
	\alakohta{3,4,5}
	\alakohta{2,3,4}
	\alakohta{1,2,3}
\end{alakohdat}
\begin{vastaus}
\begin{alakohdat}
	\alakohta{$6$}
	\alakohta{$\frac{3\sqrt{15}}{4} \approx 2,90$}
	\alakohta{Kolmion sivujen pituudet eivät voi olla 1, 2 ja 3 (kolmioepäyhtälö).}
\end{alakohdat}
\end{vastaus}
\end{tehtava}

\paragraph*{Hallitse kokonaisuus}

\begin{tehtava}
Jos kolmion $ABC$ kaikki sivut ovat pidempiä kuin kolmion $DEF$ sivut, onko kolmion $ABC$ ala välttämättä suurempi kuin $DEF$:n ala?
\begin{vastaus}
Ei. Jos esimerkiksi kolmion $ABC$ sivujen pituudet ovat kukin 8 (eli kolmion on tasasivuinen), ja kolmion $DEF$ sivujen pituudet ovat 9, 9 ja 17, kolmion $ABC$ ala on suurempi kuin kolmion $DEF$ ala.
\end{vastaus}
\end{tehtava}

\begin{tehtava}
Muuten edellinen tehtävä, mutta tiedetään lisäksi, että kolmioilla on yksi yhteinen kulma.
\begin{vastaus}
Kyllä. Jos $\alpha$ on tämä yhteinen kulma, joka on kolmion $ABC$ ja $DEF$ sivujen ja $a$ ja $b$, ja $d$ ja $e$ välissä, vastaavasti. Nyt kolmion $ABC$ ala $= ab \sin \alpha > de \sin \alpha =$ kolmion $DEF$ ala. 
\end{vastaus}
\end{tehtava}

\paragraph*{Sekalaisia tehtäviä}



\begin{tehtava}
Piste $P$ on kolmion $ABC$ sisällä. Suora $AP$ leikkaa sivun $BC$ pisteessä $D$. Todista, että
\[
\frac{BD}{CD} = \frac{|BDA|}{|CDA|} = \frac{|BDP|}{|CDP|} = \frac{|BPA|}{|CPA|}
\]
missä $|BDA|$ on kolmion $BDA$ ala jne.. Todista edelleen \termi{Cevan lause}{Cevan lause}: Jos pisteet $D$, $E$ ja $F$ ovat kolmion sivuilla $BC$, $CA$ ja $AB$ vastaavasti, suorat $AD$, $BE$ ja $CF$ kulkevat saman pisteen kautta jos ja vain jos
\[
\frac{BD\cdot CE \cdot AE}{CD \cdot EA \cdot FB}
\]
\begin{vastaus}
Vinkki: Kolmiolla $BDA$ ja $CDA$ on sama korkeus, vastaavasti kolmioilla $BDP$ ja $CDP$. Tutki verrantoja. $\Rightarrow$ Cevan lauseen suhteet voidaan esittää kolmoiden $BPA$, $APC$ ja $CPB$ alojen avulla. $\Leftarrow$ Tutki pistettä, $F'$ (janalla $AB$), jolla suorat kulkevat saman pisteen kautta ja käytä jo todistettua puolta todistaaksesi, että $F' = F$.
\end{vastaus}
\end{tehtava}
\end{tehtavasivu}

