\section*{Ympyrän tangentti}

Ympyrällä ja suoralla voi olla nolla, yksi tai kaksi yhteistä pistettä. Jos yhteisiä pisteitä on kaksi, kutsutaan suoraa sekantiksi. Mikäli yhteisiä pisteitä on yksi, kutsutaan suoraa tangentiksi.


\laatikko{
	\termi{ympyrän tangentti}{ympyrän tangentti} on suora joka leikkaa ympyrän tasan yhdessä pisteessä, toisin sanoen se sivuaa ympyrää.
	
	Ympyrän tangentti on kohtisuorassa sivuamispisteeseen piirrettyä sädettä vastaan.
	}

 Ympyrän ulkopuolisen pisteen kautta voidaan piirtää aina kaksi tangenttia.
	
	%FIXME suorakulma ei ehkä näytä oikealta, ei ole wikin dokumentaatiossa komentoa geom.suorakulma
	
	%O origo, A tangenttien leikkauspiste, B ja C sivuamispisteet
\begin{kuva}
    rajaa(minX = -1, maxX = 5)
    rajaa(minY = -1.5, maxY = 4)
    
    A=(3.414,1)
    B=(1,1)
    O =(1,0)
    C=(1.707, -0.707)
    Y = geom.ympyra(O,1,piirra = True)
    OB = geom.jana(B,O)
    OA = geom.jana(C,O)
    geom.suora(A,B)
    geom.suora(C,A)
    geom.suorakulma(A, C, O)
    geom.suorakulma(O, B , A)
\end{kuva}

\begin{esimerkki}
Ympyrän tangenttien välinen kulma on $40^\circ$. Mikä on tangenttien rajaamaa kaarta vastaavan keskuskulman suuruus?

\end{esimerkki}

\begin{tehtavasivu}

\paragraph*{Opi perusteet}

\paragraph*{Hallitse kokonaisuus}
  \begin{tehtava}
	Kaverisi on juuri tullut Dubain lomalta ja kerskuu seikkailuistaan. Hän kertoo kuinka oli Burj Khalifan terassilla ihailemassa kaunista auringonlaskua, ja kaksi ja puoli minuuttia myöhemmin todisti saman auringonlaskun tornin pohjalta! Epäilet kaverisi liioittelevan ja haluat tarkistaa hänen tarinansa. Laske kuinka korkealla kaverisi on pitänyt olla jotta hänen tarinansa olisi totta. Tiedät, että torni on ainakin 452 metriä korkea. Maan säde on noin 6400 km.
  %kuva!
	\begin{vastaus}
	381 m
  	\end{vastaus}
  \end{tehtava}
\paragraph*{Sekalaisia tehtäviä}

\end{tehtavasivu}


