\section{Pythagoraan lause}

Pythagoraan lause on eräs vanhimmista matemaattisista lauseista. Sille tunnetaan monia eri todistuksia, joista kaksi yksinkertaista esitellään seuraavaksi.

\laatikko{
	\termi{Pythagoraan lause}{Pythagoraan lause}. Olkoot suorakulmaisen kolmion kateettien pituudet $a$ ja $b$, ja hypotenuusan pituus $c$. Tällöin pätee
	\[
	a^2 + b^2 = c^2
	\] 
	}
%	\begin{center}
%\begin{kuva}
%skaalaa(2)
%	A=geom.piste(0,0)
%	B=geom.piste(0,2)
%	C=geompiste.(1,0)
%	D=geom.piste(0.1,0.1)
%	E=geom.piste(0,0.1)
%	F=geom.piste(0.1,0)
%
%	geom.jana(B,A,"$a$")
%	geom.jana(C,B,r"$c$")
%	geom.jana(A,C,"$b$")
%	geom.jana(E,D)
%	geom.jana(D,F)
%\end{kuva}
%\end{center}

Sama pätee myös käänteiseen suuntaan eli jos kolmion sivujen pituuden toteuttavat ehdon $a^2 + b^2 = c^2$, niin kolmio on suorakulmainen. Pythagoraan lause voidaan luokitella \termi{potenssiyhtälö}{potenssiyhtälöksi}. (Potenssiyhtälöitä opeteltiin ratkaisemaan edellisillä kursseilla.) Tavallisesti kahta lyhyempää sivua, kateettia (niiden pituutta) merkataan $a$:lla ja $b$:llä, ja hypotenuusan pituutta $c$:llä kuten kuvassa. Yhtälön ratkaisun kannalta olennaista on, että geometrian sovelluksissa emme hyväksy yhtälön negatiivisia juuria, koska pituudet, pinta-alat ja tilavuudet ovat positiivisia.

\textbf{Pythagoraan lauseen todistus}. Olkoot $A$, $B$ ja $C$ suorakulmaisen kolmion kärjet niin, että kulma $\angle ACB$ on suora. Olkoon piste $D$ sellainen janan $AB$ piste, jolle $\angle CDA$ on suora.

	\begin{center}
\begin{kuva}
skaalaa(3)
A = geom.piste(0, 0, "$A$", suunta = -135)
B = geom.piste(2, 0, "$B$")
C = geom.ympyranKehapiste(geom.ympyra((1, 0), 1, piirra = False), 70, "$C$", suunta = 90)

AB = geom.jana(A, B)
geom.jana(B, C)
geom.jana(C, A)

geom.suorakulma(A, C, B)

D = geom.projektio(C, AB, "$D$", suunta = -90)
geom.jana(C, D)

geom.suorakulma(C, D, A)
\end{kuva}
\end{center}

Tällöin kolmioilla $ABC$ ja $ACD$ on yhteinen kulma $\angle A$ ja molemmissa on suora kulma, joten ne ovat yhdenmuotoiset (kk). Samoin kolmioilla $ABC$ ja $CBD$ on yhteinen kulma $\angle B$ ja molemmissa on suora kulma, joten nekin ovat yhdenmuotoiset (kk)

Yhdenmuotoisuuksista saadaan, että
\[
\frac{AB}{AC} = \frac{AC}{AD}
\]
ja
\[
\frac{AB}{CB} = \frac{CB}{DB}.
\]

Ristiinkertomalla saadaan, että
\[
AB \cdot AD = AC^2
\]
ja
\[
AB \cdot DB = CB^2.
\]
Laskemalla nämä saadut yhtälöt yhteen saadaan

\begin{align*}
AC^2 + CB^2  = AB \cdot AD + AB \cdot DB \\
AC^2 + CB^2  = AB(AD + DB) \\
AC^2 + CB^2  = AB^2.
\end{align*}
Tämä on täsmälleen Pythagoraan lause. $\square $

\begin{esimerkki}
Määritä oheisen suorakulmaisen kolmion hypotenuusan pituus.
\begin{center}
\begin{kuva}
	skaalaa(1)

	A=(0,0)
	B=(0,1)
	C=(2,0)
	D=(0.1,0.1)
	E=(0,0.1)
	F=(0.1,0)

	geom.jana(B,A,"$3$")
	geom.jana(C,B,"$c$")
	geom.jana(A,C,"$4$")
	geom.jana(E,D)
	geom.jana(D,F)
\end{kuva}
\end{center}

\begin{esimratk}
Kateettien pituudet ovat $a=3$ ja $b=4$, ja tuntematonta hypotenuusaa on merkitty $c$:llä. (Sinänsä ei ole väliä, kumpaa kateettia merkataan milläkin kirjaimella.) Pythagoraan lauseesta saadaan:

\begin{align*}
a^2+b^2 &= c^2  && || \text{sijoitetaan tehtävänannon arvot kaavaan} \\
3^2+4^2 &= c^2 && || \text{sievennetään yhtälön vasenta puolta} \\
9+16 &= c^2 && \\
25 &= c^2 && || \sqrt{} \\
5 &= c &&
\end{align*}

Hyväksymme vain positiivisen juuren, koska pituudet eivät voi olla negatiivisia.

\end{esimratk}
	\begin{esimvast}
Kolmion hypotenuusan pituus on $5$ (pituusyksikköä).
	\end{esimvast}
\end{esimerkki}

\begin{esimerkki}
Taulutelevision kooksi (lävistäjäksi) on ilmoitettu mainoksessa $46,0$ tuumaa ($116,8$ cm) ja kuvasuhteeksi $16:9$. Kuinka leveä televisio on senttimetreinä?

%FIXME: KIMMON KUVITUSKUVA!
	\begin{esimratk}
Taulutelevision halkaisija, alareuna ja toinen sivu muodostavat suorakulmaisen kolmion. Kolmion hypotenuusa ($c$) on television halkaisija ja kateetit ($a$ ja $b$) alareuna ja toinen sivu. Kuvasuhteen perusteella kateettien pituuksia voidaan merkitä $16x$ ja $9x$, missä $x$ esittää pienintä sivujen yhteistä mittaa. Sivuja ei voida merkitä suoraan $16$ ja $9$, koska nämä eivät ole oikeita pituuksia vaan vain suhteita. Pythagoraan lauseesta ($c^2 = a^2 + b^2$) saadaan
\[
116,8^2 = (16x)^2 + (9x)^2
\]
\[
13\,642,24 = (256+81)x^2.
\]
\[
x^2 = \frac{13\,642,24}{337}
\]
\[
x= \sqrt{\frac{13\,642,24}{337}} \approx 6,36.
\]
Television leveys on noin $16x = 16\cdot 6,36\approx 102$\,cm.
	\end{esimratk}
	\begin{esimvast}
	Noin $102$\,cm
	\end{esimvast}
\end{esimerkki}

\begin{esimerkki}
	Maalari maalaa taloa. Hänellä on tikapuut, joiden pituus on $3,0$ metriä. Maalarin on noustava $2,5$ metrin korkeuteen voidakseen maalata katonrajan sinipunaiseksi. Kuinka lähelle seinää hänen on asetettava tikapuiden alapää?
	\begin{esimratk}
		Tikapuut ja seinä muodostavat suorakulmaisen kolmion, jonka hypotenuusa on tikapuut ja toinen kateetti on seinä. Merkitään $x$:llä tikapuiden etäisyyttä seinästä metreinä ja sovelletaan Pythagoraan lausetta.
		\begin{align*}
		x^2 + 2,5^2 &= 3,0^2 \\
		x^2 + 6,25 &= 9,0 \\
		x^2 &= 2,75 \\
		x &= \pm \sqrt{2,75} \approx \pm 1,7
		\end{align*}
		Koska $x$ on etäisyys seinästä, negatiivinen ratkaisu ei kelpaa.
		\begin{esimvast}
		Alle $1,7$ metrin päähän seinästä. 
		\end{esimvast}
	\end{esimratk}
\end{esimerkki}

%\begin{esimerkki}
%$\star$ Suorakulmion muotoisen levyn mitat ovat $230\,\text{cm}\times 250\,\text{cm}$. Mahtuuko se sisään oviaukosta, joka on $90$\,cm leveä ja $205$\,cm korkea?
%
%        Ei mahdu. Suurin mahdollinen levy, joka mahtuu ovesta sisään on Pythagoraan lauseen perusteella leveydeltään $\sqrt{90^2+205^2}\approx 224$\,cm.
%
%\end{esimerkkki}

%FIXME Kuva ois kiva esimerkkiin, piirroskuva, jossa tikapuut ois vielä kivempi, tulossa

\begin{tehtavasivu}

\paragraph*{Opi perusteet}

\paragraph*{Hallitse kokonaisuus}

	\begin{tehtava}
Äiti avasi kukkaronsa nyörejä ja ostaa paukautti suuren taulutelevision, jonka mitat ovat $300$\,cm \times $210$\,cm. Mahtuuko televisio edes sisään ulko-ovesta, jonka mitat on $200$\,cm \times $80$\,cm?
	\begin{vastaus}
	 Kyllä mahtuu. (Oven lävistäjä on suurempi kuin telkkarin korkeus.)
	\end{vastaus}
	\end{tehtava}
	
	\begin{tehtava}
Suorakulmion muotoisen vanerilevyn mitat ovat $230\,\text{cm}\times 250\,\text{cm}$. Mahtuuko se sisään oviaukosta, joka on $90$\,cm leveä ja $205$\,cm korkea?
        \begin{vastaus}
        Ei mahdu. Suurin mahdollinen levy, joka mahtuu ovesta sisään on Pythagoraan lauseen perusteella leveydeltään $\sqrt{90^2+205^2}\approx 224$\,cm.
        \end{vastaus}
\end{tehtava}

\begin{tehtava}
$\star$ Ajatellaan suorakulmaista hiekkakenttää, jonka pinta-ala on $1$ aari ($100\,\mathrm{m}^2$). Lyhyemmän ja pidemmän sivujen pituuksien suhde on $4:3$. Laske Pythagoraan lauseen avulla matka hiekkakentän kulmasta kauimmaisena olevaan kulmaan.
\begin{vastaus}
$\frac{4}{3}x^2=100$, joten $x = \sqrt{\frac{300}{4}}$. 
Hypotenuusa: $\sqrt{x^2 + (\frac{4}{3}x)^2}=\sqrt{\frac{300}{4}+\frac{16}{9}\cdot \frac{300}{4}}
=\frac{5}{3}\sqrt{\frac{300}{4}}\approx 14,4$.
\end{vastaus}
\end{tehtava}

\begin{tehtava}
Todista Pythagoraan lauseen käänteislause: Jos kolmion sivujen pituuksille $a,b$ ja $c$ pätee
\[
a^2+b^2 = c^2,
\]
niin kolmio on suorakulmainen.

\begin{vastaus}
Vinkki: Tutki suorakulmaista kolmiota, jonka kateettien pituudet ovat $a$ ja $b$ ja hyödynnä yhtenevyyttä.
\end{vastaus}
\end{tehtava}

\begin{tehtava}
Kolmion ala voidaan esittää myös suoraan kolmion sivujen pituuksien funktiona. Kolmion $ABC$ sivua $BC$ vastaavan korkeusjanan $AD$ pituus voidaan määrittää käyttämällä Pythagoraan lausetta kolmioihin $ABD$ ja $ACD$.
\begin{alakohdat}
\alakohta{Lausu sivun $BC$ pituus sivujen $AC$ ja $AB$ ja korkeusjanan $AD$ pituuksien avulla.}
\alakohta{Ratkaise korkeusjanan pituuden neliö neliöimällä yhtälö ja sieventämällä kahdesti.}
\alakohta{Nyt voit muodostaa polynomilausekkeen kolmion pinta-alan neliölle.}
\alakohta{Merkitse vielä $BC = a$, $AC = b$, $AB = c$, sekä $p = \frac{a+b+c}{2}$ ja todista kuuluisa Heronin kaava kolmion pinta-alalle
\[
A_{ABC} = \sqrt{p(p-a)(p-b)(p-c)}
\]
}
\end{alakohdat}

\begin{vastaus}
\begin{alakohdat}
\alakohta{$BC = \pm\sqrt{AC^2-AD^2}\pm\sqrt{AB^2-AD^2}$. $\pm$-merkit valitaan riippuen kolmion tyypistä.}
\alakohta{$AD^2 = \frac{4AC^2AB^2-(BC^2-AC^2-AB^2)^2}{BC^2}$}
\alakohta{$A_{ABC}^2 = \frac{4AC^2AB^2-(BC^2-AC^2-AB^2)^2}{16}$}
\alakohta{Vinkki: Käytä toistuvasti identiteettiä $x^2-y^2 = (x+y)(x-y)$ tai kerro raa'asti auki.}
\end{alakohdat}
\end{vastaus}
\end{tehtava}

\paragraph*{Sekalaisia tehtäviä}

\begin{tehtava}
	Ratkaise puuttuvat sivut
	\begin{alakohdat}
		\alakohta{Kateetit ovat $3$ ja $4$}
		\alakohta{Kateetit ovat $1$ ja $2$}
		\alakohta{Toinen kateetti, kun hypotenuusa on $\sqrt{2}$ ja toinen kateetti on $1$}
		\alakohta{Hypotenuusa on $8$, kun molemmat kateetit ovat yhtä pitkiä}
	\end{alakohdat}
	\begin{vastaus}
		\begin{alakohdat}
			\alakohta{$5$}
			\alakohta{$\sqrt{5}$}
			\alakohta{$1$}
			\alakohta{$2\sqrt{2}$}
		\end{alakohdat}
	\end{vastaus}
\end{tehtava}

\begin{tehtava}
Todista, että jos kolmion sivujen pituudet ovat $2ab$, $a^2-b^2$ ja $a^2+b^2$ ($a>b$), niin kolmio on suorakulmainen.
\end{tehtava}
\end{tehtavasivu}


