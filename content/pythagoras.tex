\section{Pythagoraan lause}

Pythagoraan lause on eräs vanhimmista matemaattisista lauseista. Sille tunnetaan monia eri
todistuksia, joista kaksi yksinkertaista esitellään seuraavaksi.

\laatikko{
	\termi{Pythagoraan lause}{Pythagoraan lause}. Olkoot suorakulmaisen kolmion kateettien
	pituudet $a$ ja $b$, ja hypotenuusan pituus $c$. Tällöin pätee
	\[
	a^2 + b^2 = c^2
	\] 
}

Sama pätee myös käänteiseen suuntaan eli jos kolmion sivujen pituuden toteuttavat ehdon $a^2 + b^2 = c^2$, niin kolmio on suorakulmainen.

\textbf{Pythagoraan lauseen todistus}. Olkoot $A$, $B$ ja $C$ suorakulmaisen kolmion
kärjet niin, että kulma $\angle ACB$ on suora. Olkoon piste $D$ sellainen janan $AB$ piste,
jolle $\angle CDA$ on suora.

\begin{kuva}
skaalaa(3)

A = geom.piste(0, 0, "$A$", suunta = -135)
B = geom.piste(2, 0, "$B$")
C = geom.ympyranKehapiste(geom.ympyra((1, 0), 1, piirra = False), 70, "$C$", suunta = 90)

AB = geom.jana(A, B)
geom.jana(B, C)
geom.jana(C, A)

geom.suorakulma(A, C, B)

D = geom.projektio(C, AB, "$D$", suunta = -90)
geom.jana(C, D)

geom.suorakulma(C, D, A)


\end{kuva}

Tällöin kolmioilla $ABC$ ja $ACD$ on yhteinen kulma $\angle A$ ja molemmissa on suora kulma, joten ne ovat yhdenmuotoiset (kk). Samoin kolmioilla $ABC$ ja $CBD$ on yhteinen kulma
$\angle B$ ja molemmissa on suora kulma, joten nekin ovat yhdenmuotoiset (kk)

Yhdenmuotoisuuksista saadaan, että
\[
\frac{AB}{AC} = \frac{AC}{AD}
\]
ja
\[
\frac{AB}{CB} = \frac{CB}{DB}.
\]

Ristiinkertomalla saadaan, että
\[
AB \cdot AD = AC^2
\]
ja
\[
AB \cdot DB = CB^2.
\]
Laskemalla nämä saadut yhtälöt yhteen saadaan

\begin{align*}
AC^2 + CB^2  = AB \cdot AD + AB \cdot DB \\
AC^2 + CB^2  = AB(AD + DB) \\
AC^2 + CB^2  = AB^2.
\end{align*}
Tämä on täsmälleen Pythagoraan lause. $\square $

\begin{esimerkki}
	Maalari maalaa taloa. Hänellä on tikapuut, joiden pituus on 3,0 metriä. Maalarin on
	noustava 2,5 metrin korkeuteen voidakseen maalata katonrajan sinipunaiseksi. Kuinka
	lähelle seinää hänen on asetettava tikapuiden alapää?
	\begin{esimratk}
		Tikapuut ja seinä muodostavat suorakulmaisen kolmion, jonka hypotenuusa on tikapuut
		ja toinen kateetti on seinä. Merkitään $x$:llä tikapuiden etäisyyttä seinästä
		metreinä ja sovelletaan Pythagoraan lausetta.
		\begin{align*}
		x^2 + 2,5^2 &= 3,0^2 \\
		x^2 + 6,25 &= 9,0 \\
		x^2 &= 2,75 \\
		x &= \pm \sqrt{2,75} \approx \pm 1,7
		\end{align*}
		Koska $x$ on etäisyys seinästä, negatiivinen ratkaisu ei kelpaa.
		\begin{esimvast}
		Alle $1,7$ metrin päähän seinästä. 
		\end{esimvast}
	\end{esimratk}
\end{esimerkki}

%FIXME Kuva ois kiva esimerkkiin, piirroskuva, jossa tikapuut ois vielä kivempi, tulossa

\begin{tehtavasivu}

\paragraph*{Opi perusteet}

\paragraph*{Hallitse kokonaisuus}

\begin{tehtava}
Todista Pythagoraan lauseen käänteislause: Jos kolmion sivujen pituuksille $a,b$ ja $c$ pätee
\[
a^2+b^2 = c^2,
\]
niin kolmio on suorakulmainen.

\begin{vastaus}
Vinkki: Tutki suorakulmaista kolmiota, jonka kateettien pituudet ovat $a$ ja $b$ ja hyödynnä yhtenevyyttä.
\end{vastaus}
\end{tehtava}

\begin{tehtava}
Kolmion ala voidaan esittää myös suoraan kolmion sivujen pituuksien funktiona. Kolmion $ABC$ sivua $BC$ vastaavan korkeusjanan $AD$ pituus voidaan määrittää käyttämällä Pythagoraan lausetta kolmioihin $ABD$ ja $ACD$.
\begin{alakohdat}
\alakohta{Lausu sivun $BC$ pituus sivujen $AC$ ja $AB$ ja korkeusjanan $AD$ pituuksien avulla.}
\alakohta{Ratkaise korkeusjanan pituuden neliö neliöimällä yhtälö ja sieventämällä kahdesti.}
\alakohta{Nyt voit muodostaa polynomilausekkeen kolmion pinta-alan neliölle.}
\alakohta{Merkitse vielä $BC = a$, $AC = b$, $AB = c$, sekä $p = \frac{a+b+c}{2}$ ja todista kuuluisa Heronin kaava kolmion pinta-alalle
\[
A_{ABC} = \sqrt{p(p-a)(p-b)(p-c)}
\]
}
\end{alakohdat}

\begin{vastaus}
\begin{alakohdat}
\alakohta{$BC = \pm\sqrt{AC^2-AD^2}\pm\sqrt{AB^2-AD^2}$. $\pm$-merkit valitaan riippuen kolmion tyypistä.}
\alakohta{$AD^2 = \frac{4AC^2AB^2-(BC^2-AC^2-AB^2)^2}{BC^2}$}
\alakohta{$A_{ABC}^2 = \frac{4AC^2AB^2-(BC^2-AC^2-AB^2)^2}{16}$}
\alakohta{Vinkki: Käytä toistuvasti identiteettiä $x^2-y^2 = (x+y)(x-y)$ tai kerro raa'asti auki.}
\end{alakohdat}
\end{vastaus}
\end{tehtava}

\paragraph*{Sekalaisia tehtäviä}

\begin{tehtava}
	Ratkaise puuttuvat sivut
	\begin{alakohdat}
		\alakohta{Kateetit ovat $3$ ja $4$}
		\alakohta{Kateetit ovat $1$ ja $2$}
		\alakohta{Toinen kateetti, kun hypotenuusa on $\sqrt{2}$ ja toinen kateetti on $1$}
		\alakohta{Hypotenuusa on $8$, kun molemmat kateetit ovat yhtä pitkiä}
	\end{alakohdat}
	\begin{vastaus}
		\begin{alakohdat}
			\alakohta{$5$}
			\alakohta{$\sqrt{5}$}
			\alakohta{$1$}
			\alakohta{$2\sqrt{2}$}
		\end{alakohdat}
	\end{vastaus}
\end{tehtava}

\begin{tehtava}
Todista, että jos kolmion sivujen pituudet ovat $2ab$, $a^2-b^2$ ja $a^2+b^2$ ($a>b$), niin kolmio on suorakulmainen.
\end{tehtava}
\end{tehtavasivu}


