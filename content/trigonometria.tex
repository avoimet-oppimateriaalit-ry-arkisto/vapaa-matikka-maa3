
\section*{Trigonometria}

Tarkastellaan kahta suorakulmaista kolmiota, joilla on yhtä suuri terävä kulma $\alpha$.
Silloin kolmiot ovat keskenään yhdenmuotoisia, sillä niillä on kaksi yhtä suurta vastinkulmaa.
Yhdenmuotoisuuden nojalla sivujen keskinäiset suhteet ovat siis täsmälleen samoja kaikilla
suorakulmaisilla kolmioilla, joilla on samansuuruinen terävä kulma. Näille suhteet on
nimetty seuraavasti.

\laatikko{
\termi{sini}{Sini}: $\sin \alpha = \frac{\textrm{kulman $\alpha$ vastaisen kateetin pituus}}
{\textrm{hypotenuusan pituus}}$ \\
\termi{kosini}{Kosini}: $\cos \alpha = \frac{\textrm{kulman $\alpha$ viereisen kateetin pituus}}
{\textrm{hypotenuusan pituus}}$ \\
\termi{tangentti}{Tangentti}: $\tan \alpha = \frac{\textrm{kulman $\alpha$ vastaisen kateetin pituus}}
{\textrm{kulman $\alpha$ viereisen kateetin pituus}}$
}

\begin{esimerkki}
\begin{center}
\begin{kuva}
	A = (4, 3)
	B = (0, 0)
	C = (4, 0)
	geom.jana(A, B, "$c$")
	geom.jana(B, C, "$a$")
	geom.jana(C, A, "$b$")
	geom.suorakulma(A, C, B)
	geom.kulma(C, B, A, r"$\alpha$")
\end{kuva}
\end{center}
Kuvan kolmiossa pätee
\[
\begin{array}{ccc}
\displaystyle\sin \alpha = \frac{b}{c}, &
\displaystyle\cos \alpha = \frac{a}{c}, &
\displaystyle\tan \alpha = \frac{b}{a}.
\end{array}
\]
\end{esimerkki}

...yksikköympyrä, tylpän kulman siniä ja kosinia tarvitaan...


%mihin kohtaan sijoitetaan?
Koordinaattiakselit jakavat koordinaatiston neljään osaan. Näitä osia kutusutaan neljänneksiksi ja niitä merkitään roomalaisin numeroin (I, II, III ja IV) alkaen positiivisesta $x$-akselista positiiviseen kiertosuuntaan (vastapäivään).

\subsection{Yksikköympyrä}

%vain tapauksissa 0-180 astetta, loput kulmat kurssilla 9

Yksikköympyräksi kutsutaan koordinaatistoon pirrettyä ympyrää, jonka keskipiste on origossa ja säde yksi.

%\FIXME yksikköympyrässä käytetään termiä kehäpiste. Termi esitellään vasta myöhemmin. Mikä on esitysjärjestys?
Kulman kärki sijoitetaan origoon ja oikea kylki positiiviselle $x$-akselille.

%pitääkö mainita?
Kulman suuruutta mitataan positiiviseen kiertosuuntaan.

% 
% O=geom.piste(0,0)
%         A=geom.piste(0.5, 0.866)
%         B=geom.piste(0, 0.866)
%         C=geom.piste(-0.5, 0.866)
%         D=geom.piste(1,0)

\begin{center}
\begin{kuva}
        kuvaaja.pohja(-1.5, 1.5, -1.5, 1.5)
        O=(0,0)
        A=geom.piste(0.5, 0.866, r"keh\"apiste")
        B=(0, 0.866)
        C=(-0.5, 0.866)
        D=(1,0)
        geom.jana(O, A)
        geom.kulma(D, O, A, r"$\alpha$")
        geom.ympyra(O,1)
\end{kuva}
\end{center}


%sini
\begin{center}
\begin{kuva}
        kuvaaja.pohja(-1.5, 1.5, -1.5, 1.5)
        O=(0,0)
        A=geom.piste(0.5, 0.866)
        B=(0, 0.866)
        C=(-0.5, 0.866)
        D=(1,0)
        geom.jana(O, A)
        geom.kulma(D, O, A, r"$\alpha$")
        geom.ympyra(O,1)
        geom.jana(A,B)
\end{kuva}
\end{center}

%kosini
\begin{center}
\begin{kuva}
        kuvaaja.pohja(-1.5, 1.5, -1.5, 1.5)
        O=(0,0)
        A=geom.piste(0.5, 0.866)
        B=(0, 0.866)
        C=(-0.5, 0.866)
        D=(1,0)
        E=(0.5,0)
        geom.jana(O, A)
        geom.kulma(D, O, A, r"$\alpha$")
        geom.ympyra(O,1)
        geom.jana(A,E)
\end{kuva}
\end{center}

%FIXME \todo{olisi kiva, jos kuvaajapohjasta olisi ''millimetripaperiversio''}

\begin{tehtavasivu}
\paragraph*{Opi perusteet}

  \begin{tehtava}
    Olet opiskellut myöhään yöhön ja toivot saavasi nukkua myöhään, mutta aamuaurinko paistaa sisään ja herättää sinut. Mahdut siirtymään sängylläsi 40 sentin päähän ikkunasta, mutta riittääkö se pelastukseen?
    
    Auringon säteen ja maanpinnan välinen kulma on $40^\circ$ ja sänky on 30 cm ikkunan alareunan alla.
    \begin{vastaus}
    Riittää. Aurinko paistaa noin 35 cm päähän ikkunasta.
    \end{vastaus}
  \end{tehtava}

\paragraph*{Hallitse kokonaisuus}

\paragraph*{Sekalaisia tehtäviä}
\end{tehtavasivu}

