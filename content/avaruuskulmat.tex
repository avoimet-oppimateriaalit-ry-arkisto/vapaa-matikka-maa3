\section*{Avaruuskulmat}

%\subsection*{Tason ja suoran välinen kulma}
Tason ja suoran leikatessa toisensa muodostuu kaksi kulmaa, joista pienempää kutsutaan tason ja suoran väliseksi kulmaksi. Jos kulmat ovat yhtä suuret, valitaan niistä toinen, jonka suuruus on 90 astetta. %täydennä asteen merkki?

%\subsection*{Kahden tason välinen kulma}
Kahden tason leikatessa toisensa niiden välille muodostuu kaksi kulmaa, ja kuten tason ja suoran välisessä tapauksessa valitaan tässäkin kahden tason väliseksi kulmaksi pienempi tai suora kulma.

\begin{tehtavasivu}

\paragraph*{Opi perusteet}

\begin{tehtava}
Huoneen seinä ja lattia ovat kohtisuorassa toisiaan vastaan. Seinässä on ovi, jonka leveys on 80 cm ja korkeus 210 cm. Ovi avataan niin, että se on $40^\circ$ kulmassa seinää vasten.
Kuinka kaukana oven aukeava yläkulma on silloin
\begin{alakohdat}
\alakohta{oviaukon yläkulmasta?}
\alakohta{seinästä?}
\alakohta{oviaukon alakulmasta?}
\end{alakohdat}
\begin{vastaus}
\begin{alakohdat}
\alakohta{55 cm}
\alakohta{51 cm}
\alakohta{220 cm}
\end{alakohdat}
\end{vastaus}
\end{tehtava}

\paragraph*{Hallitse kokonaisuus}

\paragraph*{Sekalaisia tehtäviä}

\end{tehtavasivu}