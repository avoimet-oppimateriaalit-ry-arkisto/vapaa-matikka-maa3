\section*{Avaruuskappaleet}
Geometriassa tutkitaan tasokuvioiden lisäksi myös erilaisia kolmiulotteisia kappaleita eli avaruuskappaleita. Näistä yleisimpiä ovat \termi{särmiö}{särmiö}, \termi{pallo}{pallo}, \termi{lieriö}{lieriö} ja \termi{kartio}{kartio}. Avaruuskappaleille voidaan laskea pinta-alan lisäksi myös tilavuus.

%Varmaan kannattaa jakaa tämä alalukuihin...
%Minne kaikkialle laitetaan kuvia?
%Jotakin siitä, että kartion ja lieriön tilavuus ei ole riippuvainen vaipan sivujanan pituudesta (esim vinolla ja suoralla lieriöllä voi olla sama tilavuus mutta eri vaipan pinta-ala)

Lieriöpinta muodostuu, kun suoraa kuljetetaan umpinaista käyrää pitkin. Kun lieriöpinta katkaistaan kahdella samansuuntaisella tasolla, tasojen väliin jää avaruuskappale, jota kutsutaan \termi{lieriö}{lieriöksi}. Lieriön osia kutsutaan pohjaksi ja vaipaksi.

Merkitään lieriön pohjan pinta-alaa kirjaimella $A_p$ ja lieriön korkeutta kirjaimella $h$.

\laatikko{
Lieriön tilavuus $V$ saadaan kaavasta
$$V= A_p \cdot h.$$
}

Lieriötä, jonka pohja on ympyrä, kutsutaan \textbf{ympyrälieriöksi}. Jos lieriön vaippa on kohtisuorassa pohjaa vastaan, lieriötä kutsutaan \textbf{suoraksi lieriöksi}. Merkitään lieriön pohjaympyrän sädettä kirjaimella $r$.

\laatikko{
Suoran ympyrälieriön tilavuus $V$ saadaan kaavasta
$$V=\pi \cdot r^2 h.$$
}

Suoran ympyrälieriön kokonaispinta-ala saadaan laskemalla yhteen vaipan pinta-ala ja kaksi kertaa pohjan pinta-ala.

\laatikko{
Suoran ympyrälieriön vaipan pinta-ala $A_v$ saadaan kaavasta
$$A_v=2\pi \cdot r \cdot h.$$
Suoran ympyrälieriön kokonaispinta-ala on
$$A=2\pi \cdot r \cdot h + 2\pi \cdot r^2.$$
}

\begin{esimerkki}
Suoran ympyrälieriön muotoisen purkin korkeus on 30 cm ja sen pohjan säde on 8 cm. Kuinka monta litraa kirsikkahilloa purkkiin mahtuu?

\textbf{Ratkaisu.}

Purkin tilavuus on
$$A=\pi\cdot 8^2 \cdot 30 \text{cm}^3 = 15360 \cdot \pi \text{cm}^3 \approx 4800 \text{cm}^3,$$
joten purkkiin mahtuu 4,8 litraa hilloa.
\end{esimerkki}

\termi{särmiö}{Särmiö} on lieriö, jonka pohja on monikulmio ja sivutahkot suunnikkaita.
Tärkein ja tutuin särmiö on \textbf{suorakulmainen särmiö}.
Särmiötä kutsutaan suorakulmaiseksi särmiöksi, jos sen kaikki tahkot ovat suorakulmioita. Jos suorakulmaisen särmiön kaikki tahkot ovat neliöitä, kappaletta kutsutaan \textbf{kuutioksi}.

Merkitään suorakulmaisen särmiön erisuuntaisia särmiä kirjaimilla $a$, $b$ ja $c$. 

\laatikko{
Suorakulmaisen särmiön tilavuus $A$ saadaan kaavasta
$$A=a \cdot b \cdot c.$$
}


\termi{kartio}{Kartio} on avaruuskappale, joka syntyy, kun puolisuoran päätepiste pidetään paikoillaan ja puolisuoraa kuljetetaan pitkin suljettua käyrää. Yhtä hyvin kartion voi ajatella olevan lieriö, jonka toinen pohja on supistunut yhdeksi pisteeksi.

Kartion tilavuus on kolmasosa vastaavan lieriön tilavuudesta. Merkitään kartion pohjan alaa kirjaimella $A_p$ ja kartion korkeutta kirjaimella h.

\laatikko{
Kartion tilavuus V saadaan kaavasta
$$v= \frac{1}{3} \cdot A_p \cdot h.$$
}

Kartiota sanotaan \textbf{ympyräkartioksi}, jos sen pohja on ympyrä. Jos pohjaympyrän keskipisteen ja kartion huipun kautta kulkeva suora on kohtisuorassa pohjaympyrää vastaan, ympyräkartiota kutsutaan \textbf{suoraksi ympyräkartioksi}.

Merkitään kartion pohjaympyrän sädettä kirjaimella r, kartion korkeutta kirjaimella h ja kartion sivujanan pituutta kirjaimella s.

\laatikko{
Ympyräkartion tilavuus V saadaan kaavasta
$$V=\frac{1}{3} \cdot \pi \cdot r^2 \cdot h.$$
Suoran ympyräkartion vaipan pinta-ala $A_v$ saadaan kaavasta
$$A_v=\pi \cdot r \cdot s.$$
}


\begin{esimerkki}
Essi haluaa tehdä 100 nekkua ympyräkartion muotoisella muotilla. Nekkumuotin korkeus on 7cm, ja sen pohjaympyrän säde on 2,5cm. Litraan nekkuseosta tarvitaan 60 grammaa mantelirouhetta. Kuinka paljon mantelirouhetta Essi tarvitsee nekkuja varten?

\textbf{Ratkaisu.}

Yhden nekkumuotin tilavuus on $\frac{1}{3} \cdot \pi \cdot 2,5^2 \cdot 7 \text{cm}^2 \approx 45,815\text{cm}^2.$ Sadan nekkumuotin yhteistilavuus on siis noin $45,815\text{cm}^2 \cdot 100 = 4581,5 {cm}^2=4,5815 l$. Koska yhteen litraan seosta tarvitaan 60 grammaa mantelirouhetta, niin sataa nekkua varten tarvitaan $4,5815 \cdot 60 g = 274,89g \approx 270 g$ mantelirouhetta. Siis Essi tarvitsee noin 270 grammaa mantelirouhetta.

\end{esimerkki}

\termi{pallo}{Pallo} on pistejoukko, jonka kaikki pisteet ovat yhtä kaukana pallon keskipisteestä. Pallon pisteiden etäisyyttä keskipisteestä kutsutaan pallon \termi{säde}{säteeksi}. Sädettä merkitään tavallisimmin kirjaimella $r$. Pallon pinta-alan ja tilavuuden selvittämiseksi riittää tietää pallon säde.

\laatikko{
Pallon pinta-ala $A$ saadaan kaavasta
$$A = 4 \cdot \pi \cdot r^2.$$

Pallon tilavuus $V$ saadaan kaavasta
$$\frac{4}{3} \cdot \pi \cdot r^3.$$
}

\begin{esimerkki}
Torin joulukuuseen tarvitaan 40 kultaista joulupalloa. Jokaisen pallon säde on 3 cm. Kuinka paljon kultamaalia tarvitaan, kun maalin riittoisuus on $6~\textnormal{m}^2$/l?

\textbf{Ratkaisu.}

Yhden pallon pinta-ala on $A= 4 \cdot \pi \cdot 3^2 \text{cm}^2=36\cdot \pi \text{cm}^2$. Kaikkien 40 pallon yhteispinta-ala on siis
$$40 \cdot 36 \cdot \pi \text{cm}^2 = 1440 \cdot \pi \text{cm}^2 \approx 4523 \text{cm}^2 =0,4523 \text{m}^2.$$
Koska maalin riittoisuus on $6~\textnormal{m}^2$/l, tarvittavan maalin määrä on $\frac{0,4523}{6}\text{l}=0,7538333 \cdots \text{l} \approx 0,8\text{l}$.
Siis kultamaalia tarvitaan 0,8 litraa.
\end{esimerkki}

\begin{tehtavasivu}

\begin{tehtava}
Kultakalaparvi tarvitsee akvaarion, johon mahtuu vähintään 100 litraa vettä kutakin kalaa kohti. Jos haluaa pitää kymmentä kultakalaa pallonmuotoisessa kalamaljassa, kuinka suuri tämän kalamaljan säteen täytyy vähintään olla?
\begin{vastaus}
62cm
\end{vastaus}
\end{tehtava}

\begin{tehtava}
Oletetaan, että litra heliumia jaksaa kannatella yhden gramman verran painoa.
\begin{enumerate}[a)]
\item Kuinka monta grammaa jaksaa kannatella heliumpallo, jonka säde on 20cm?
\item Kaisa painaa 70kg ja myy heliumpalloja, joiden säde on 20cm. Kuinka monen pallon kimppu riittää kannattelemaan Kaisan painoa?
\end{enumerate}
\begin{vastaus}
\begin{enumerate}[a)]
\item 34 grammaa
\item 2089 heliumpalloa
\end{enumerate}
\end{vastaus}
\end{tehtava}

\begin{tehtava}
Tennispallon halkaisija on noin 7cm. Tennispalloja säilytetään pinossa lieriön muotoisessa, pahvisessa kotelossa. Kuinka paljon pahvia tarvitaan koteloon, johon mahtuu 5 tennispalloa? Saumoja ei tarvitse ottaa huomioon.
\begin{vastaus}
$850~\textnormal{cm}^2$
\end{vastaus}
\end{tehtava}

\begin{tehtava}
Kartioita ovat hatut, jäätelötötteröt, liikenteenohjauskartiot, megafonit, peitsenkärjet, mitä näitä nyt on
\begin{vastaus}
...
\end{vastaus}
\end{tehtava}

\begin{tehtava}
Osoita Pythagoraan lauseen yleistys kolmeen ulottuvuuteen: suorakulmaisen särmiön lävistäjän pituudelle $d$ pätee
$a^2+b^2+c^2=d^2$, missä $a$, $b$ ja $c$ ovat särmiön sivujen pituudet.
\end{tehtava}

\end{tehtavasivu}
