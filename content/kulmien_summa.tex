\section*{Kolmion kulmien summa}

Minkä tahansa kolmion kulmien summa on $ 180^{\circ}$ eli yhtä suuri kuin oikokulma.

Tämän usein käytetyn ominaisuuden todistaminen on kohtuullisen yksinkertaista aiemmin esitettyjen tulosten perusteella.

Kun kuvan mukaisen kolmion kärjen $C$ kautta asetetaan kulkemaan sivun $AB$ kanssa yhden suuntainen suora $l$.



%	geom.suora(A, B)
%	geom.suora(C, B)
%	geom.suora(C, A)
%pitäisikö olla suora, kun samankohtaiset kulmat on määritelty suorille?

\begin{kuva}{kolmionkulmiensumma}
	rajaa(minX = -1, maxX = 6)
	A = geom.piste(3, 4,"$A$", suunta = 315)
	B = geom.piste(-1, 4, "$B$", suunta = 225)
	C = geom.piste(0, 0, "$C$", suunta = 270)
	D = (-2, 0)
	E = (6, 0)
	geom.suora(E, D, "$l$")
	geom.jana(B, C)
	geom.jana(B, A)
	geom.jana(A, C)
	geom.kulma(C, B, A, r"$\beta$")
	geom.kulma(B, A, C, r"$\alpha$")
	geom.kulma(A, C, B, r"$\gamma$")
	geom.kulma(B, C, D, r"$\beta'$")
	geom.kulma(E, C, A, r"$\alpha'$")
\end{kuva}

Suorat $l$ ja $AB$ ovat keskenään yhdensuuntaiset. Tällöin kulmat $\alpha$ ja $\alpha'$ ovat samankohtaisina kulmina yhtä suuret. Sama pätee myös kulmille $\beta$ ja $\beta'$.

Kuvan alaosan oikokulma voidaan lausua
\[\beta' + \gamma + \alpha'=\alpha' + \beta' + \gamma=\alpha+ \beta + \gamma=180^\circ\]

eli on todistettu, että kolmion kulmien summa on $180^\circ$.
\qed 



 