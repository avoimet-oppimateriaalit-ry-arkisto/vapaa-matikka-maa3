%%Ylioppilastehtäviä koonnut lähtien uusimmasta (K2014) T: Aleksi Sipola




\subsubsection*{Lyhyen oppimäärän tehtäviä}


\begin{tehtava}
	(K2014/4)  Kuution särmän pituus puolittuu. Kuinka monta prosenttia pienenee kuution
	\begin{alakohdat}
\alakohta{tilavuus?}
\alakohta{sivutahkojen yhteenlaskettu pinta-ala?}
\end{alakohdat}
						
						\begin{vastaus}
						\begin{alakohdat}
						 \alakohta{$87,5\%$}
						 \alakohta{$75\%$}
						\end{alakohdat}
						  
				\end{vastaus}
						\end{tehtava}
					
					
					
					
						
						


\begin{tehtava}
HUOM. TARVII KUVAN 
(K2014/7)Seitsemän mäntytukkia sidotaan vaijerilla alla olevan poikkileikkauskuvion mukaisesti. Kuinka paljon vaijeria tarvitaan yhteen kierrokseen? Jokaisen tukin halkaisija on $20 cm$.
Anna vastaus senttimetrin tarkkuudella. 

\begin{vastaus}
 $183 cm$
\end{vastaus}
\end{tehtava}

\begin{tehtava}(S2013/3) Tasakylkisen kolmion kylki on $90 m$ ja kanta $40 m$.
\begin{alakohdat}
\alakohta{Laske kolmion huippukulma asteen tarkkuudella}
\alakohta{Laske kolmion pinta-ala neliömetrin tarkkuudella}
 \end{alakohdat}
    \begin{vastaus}
    \begin{alakohdat}
	\alakohta{$26\circ$}
	\alakohta{$1755 m^2$}
   \end{alakohdat}
 \end{vastaus}
\end{tehtava}

\begin{tehtava}
HUOM. TARVII KUVAN 
(S2013/4) Kuvan kaari-ikkunassa on lasin tukena rimoja. Kuinka paljon rimaa tarvitaan kuvan mukaiseen kaari-ikkunaan, kun $x=20 cm$ ja $y=40 cm$? Rimaa käytetään kaikkiin kuvion janoihin ja puoliympyröiden kaariin. Anna vastaus senttimetrin tarkkuudella.

\begin{vastaus}
 $noin 988 cm$
\end{vastaus}
\end{tehtava}


\subsubsection*{Pitkän oppimäärän tehtäviä}

\begin{tehtava} 
HUOM. TARVII KUVAN 
(K2014/*14) Erään tarinan mukaan ihmiskunta kokeili liikkumista säännöllisten monikulmioiden avulla,
ennen kuin pyörä keksittiin.
\begin{alakohdat}

\alakohta{Tasasivuinen kolmio kiertyy oikealle kuvion mukaisesti, kunnes kärki \textit{A} osuu uudelleen alustaan. Kärki \textit{A} piirtää kuvion mukaisen käyrän. Laske käyrän pituus, kun kolmion piiri on \textit{p}.
(2 p.) }
\alakohta{Hahmottele vastaavat käyrät neliön ja kuusikulmion tapauksessa. Kummassakin tapauksessa monikulmio kiertyy niin monta kertaa, että vasemmalla alhaalla oleva kärki osuu uudelleen alustaan. (2 p.)}
\alakohta{Laske b-kohdan käyrän pituus neliölle, jonka piiri on \textit{p}.
(2 p.)}
\alakohta{Laske b-kohdan käyrän pituus kuusikulmiolle, jonka piiri on \textit{p}.
(3 p.)}
\end{alakohdat}

				\begin{vastaus}
				\begin{alakohdat}
				 \alakohta{$\frac{4\pi}{9}\cdot{p}$}
				 \alakohta{ Tähän kuva kanssa T: aleksi sipola}
				 \alakohta{$\frac{(2+\sqrt{2})}{8}p$}
				 \alakohta{$\frac{(2+2\sqrt{3})\pi}{9}p$}
				\end{alakohdat}

				\end{vastaus}
\end{tehtava}
