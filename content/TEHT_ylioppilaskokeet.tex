%%Ylioppilastehtäviä koonnut lähtien uusimmasta (K2014) T: Aleksi Sipola




\subsubsection*{Lyhyen oppimäärän tehtäviä}


\begin{tehtava}
	(K2014/4)  Kuution särmän pituus puolittuu. Kuinka monta prosenttia pienenee kuution
	\begin{alakohdat}
	  \alakohta{tilavuus?}
	  \alakohta{sivutahkojen yhteenlaskettu pinta-ala?}
	\end{alakohdat}
						
						\begin{vastaus}
						\begin{alakohdat}
						 \alakohta{$87,5\%$}
						 \alakohta{$75\%$}
						\end{alakohdat}					  
	\end{vastaus}
	\end{tehtava}

\begin{tehtava}
HUOM. TARVII KUVAN 
(K2014/7)Seitsemän mäntytukkia sidotaan vaijerilla alla olevan poikkileikkauskuvion mukaisesti. Kuinka paljon vaijeria tarvitaan yhteen kierrokseen? Jokaisen tukin halkaisija on $20 cm$.
Anna vastaus senttimetrin tarkkuudella. 
  \begin{vastaus}
  $183 cm$
  \end{vastaus}
\end{tehtava}

\begin{tehtava}(S2013/3) Tasakylkisen kolmion kylki on $90 m$ ja kanta $40 m$.
\begin{alakohdat}
\alakohta{Laske kolmion huippukulma asteen tarkkuudella}
\alakohta{Laske kolmion pinta-ala neliömetrin tarkkuudella}
 \end{alakohdat}
    \begin{vastaus}
    \begin{alakohdat}
	\alakohta{$26\circ$}
	\alakohta{$1755 m^2$}
   \end{alakohdat}
 \end{vastaus}
\end{tehtava}

\begin{tehtava}
HUOM. TARVII KUVAN 
(S2013/4) Kuvan kaari-ikkunassa on lasin tukena rimoja. Kuinka paljon rimaa tarvitaan kuvan mukaiseen kaari-ikkunaan, kun $x=20 cm$ ja $y=40 cm$? Rimaa käytetään kaikkiin kuvion janoihin ja puoliympyröiden kaariin. Anna vastaus senttimetrin tarkkuudella.
  \begin{vastaus}
  $noin 988 cm$
  \end{vastaus}
\end{tehtava}

\begin{tehtava}
HUOM. TARVII KUVAN 
(K2013/5) Tähtiharrastaja katselee yöllisiä tähdenlentoja pihalla, joka sijaitsee kahden kerrostalon välissä kuvan mukaisesti. Talojen korkeudet ovat $39 m$ ja $26 m$. Kuinka kaukana korkeammasta talosta molempiin suuntiin avautuu yhtä suuri kulma $\alpha$ maanpinnan tasosta katsottuna? 
  \begin{vastaus}
  $30 m$
  \end{vastaus}
\end{tehtava}

\begin{tehtava}
(K2013/6) Tennispalloja myydään suoran ympyrälieriön muotoisessa pakkauksessa, johon mahtuu neljä palloa tiiviisti päällekäin pakattuna. Tennispallon halkaisija on $6,68 cm$. Kuinka monta prosenttia pakkauksen tilavuudesta pallot täyttävät? Anna vastaus prosentin tarkkuudella.
  \begin{vastaus}
  $67\%$
  \end{vastaus}
\end{tehtava}

\begin{tehtava}(K2013/9) Neliön piiri on yhtä pitkä kuin ympyrän kehä.
\begin{alakohdat}
\alakohta{Kuinka monta prosenttia neliön pinta-ala on pienempi kuin ympyrän pinta-ala?}
\alakohta{Kuinka monta prosenttia ympyrän pinta-ala on suurempi kuin neliön pinta-ala?}
Anna vastaukset prosentin kymmenesosan tarkkuudella.
 \end{alakohdat}
    \begin{vastaus}
    \begin{alakohdat}
	\alakohta{$21,5\%$}
	\alakohta{$27,3\%$}
   \end{alakohdat}
 \end{vastaus}
\end{tehtava}

\begin{tehtava}
(S2012/2b) Suorakulmaisen kolmion hypotenuusan pituus on $4,9 m$ ja kateetin pituus $2,3 m$. Laske toisen kateetin pituus $0,1$ metrin tarkkuudella.
  \begin{vastaus}
  Noin $4,3267 m$
  \end{vastaus}
\end{tehtava}

\begin{tehtava}
(S2012/10)Suoran ympyräkartion sisällä on suora ympyrälierö, jonka pohja on kartion pohjalla ja yläreuna sivuaa kartion vaippaa. Lieriön pohjan halkaisija on yhtä suuri kuin sen korkeus. Toisaalta lieriön pohjan halkaisija on puolet kartion pohjan halkaisijasta. Kuinka monta prosenttia lieriön tilavuus on kartion tilavuudesta? Anna vastaus prosentin kymmenesosan tarkkuudella.
  \begin{vastaus}
  Noin $37,5\% m$
  \end{vastaus}
\end{tehtava}

%%TÄSSÄ ON SEURAAVAKSI PITKÄÄ MATEMATIIKKAA%%TÄSSÄ ON SEURAAVAKSI PITKÄÄ MATEMATIIKKAA%%TÄSSÄ ON SEURAAVAKSI PITKÄÄ MATEMATIIKKAA
%%TÄSSÄ ON SEURAAVAKSI PITKÄÄ MATEMATIIKKAA
%%TÄSSÄ ON SEURAAVAKSI PITKÄÄ MATEMATIIKKAA%%TÄSSÄ ON SEURAAVAKSI PITKÄÄ MATEMATIIKKAA%%TÄSSÄ ON SEURAAVAKSI PITKÄÄ MATEMATIIKKAA

\subsubsection*{Pitkän oppimäärän tehtäviä}

\begin{tehtava} 
HUOM. TARVII KUVAN 
(K2014/*14) Erään tarinan mukaan ihmiskunta kokeili liikkumista säännöllisten monikulmioiden avulla,
ennen kuin pyörä keksittiin.
  \begin{alakohdat}
  
\alakohta{Tasasivuinen kolmio kiertyy oikealle kuvion mukaisesti, kunnes kärki \textit{A} osuu uudelleen alustaan. Kärki \textit{A} piirtää kuvion mukaisen käyrän. Laske käyrän pituus, kun kolmion piiri on \textit{p}.
(2 p.) }
\alakohta{Hahmottele vastaavat käyrät neliön ja kuusikulmion tapauksessa. Kummassakin tapauksessa monikulmio kiertyy niin monta kertaa, että vasemmalla alhaalla oleva kärki osuu uudelleen alustaan. (2 p.)}
\alakohta{Laske b-kohdan käyrän pituus neliölle, jonka piiri on \textit{p}.
(2 p.)}
\alakohta{Laske b-kohdan käyrän pituus kuusikulmiolle, jonka piiri on \textit{p}.
(3 p.)}
  \end{alakohdat}

				\begin{vastaus}
				\begin{alakohdat}
				 \alakohta{$\frac{4\pi}{9}\cdot{p}$}
				 \alakohta{ Tähän kuva kanssa}
				 \alakohta{$\frac{(2+\sqrt{2})}{8}p$}
				 \alakohta{$\frac{(2+2\sqrt{3})\pi}{9}p$}
				\end{alakohdat}

				\end{vastaus}
\end{tehtava}

 \begin{tehtava} 
HUOM. TARVII KUVAN 
(S2013/6) Kolmion $ABC$ kulman $C$ puolittaja leikkaa sivun $AB$ pisteessä $D$. Pisteiden välisille etäisyyksille on voimassa $CD=6$, $AD=4$ ja $DB=3$. Määritä kolmion sivujen $AC$ ja $BC$ pituuksien tarkat arvot.
  \begin{vastaus}
  $AC=8$ ja $BC=6$
  \end{vastaus}
\end{tehtava}

\begin{tehtava} 
(S2013/10) Pöydällä on kolme samankokoista palloa, joista kukin koskettaa kahta muuta. Niiden päälle asetetaan neljäs samanlainen pallo, joka koskettaa kaikkia kolmea alkuperäistä palloa. Mikä on rakennelman korkeus? Anna vastauksena tarkka arvo pallojen säteen avulla lausuttuna.
  \begin{vastaus}
  $\left(\sqrt{\frac83}+2\right)r$
  \end{vastaus}
\end{tehtava}

\begin{tehtava} 
(K2013/4) Laske oheisen kuvan suorakulmaisen kolmion $ABC$ pinta-alan tarkka arvo.
KUVA TARVITAAN
  \begin{vastaus}
  $(5\sqrt{21}$
  \end{vastaus}
\end{tehtava}

\begin{tehtava} 
(K2013/10) Oheisen kuution särmän pituus on 2. Sen sisällä on vaaleanpunainen pallo, joka sivuaa jokaista kuution tahkoa. Kuution yhdessä kulmassa on pienempi sininen pallo, joka sivuaa suurta palloa ja kolmea kuution tahkoa kuvion mukaisesti. Laske sinisen pallon säteen tarkka arvo.
KUVA TARVITAAN
  \begin{vastaus}
  Pienen pallon säde on $r=\frac{\sqrt{3}-1}{\sqrt{3}+1}$
  \end{vastaus}
\end{tehtava}

\begin{tehtava} 
HUOM. TARVII KUVAN 
(S2013/4) 
  \begin{alakohdat}
    \alakohta{Olkoon $\alpha \in \left[\pi, \frac{3 \pi}{2}\right]$ sellainen kulma, että $\cos\alpha=-\frac13$. Määritä lukujen $\sin\alpha$ ja $\tan\alpha$ tarkat arvot.}
    \alakohta{Laske oheisessa kuvassa olevan kolmion sivun pituuden $a$ tarkka arvo ja kaksidesimaalinen likiarvo.}

  \end{alakohdat}
				\begin{vastaus}
				\begin{alakohdat}
				 \alakohta{$\sin\alpha=-\frac{2\sqrt{2}}{3}$ ja $\tan\alpha=2\sqrt{2}$}
				 \alakohta{$a\approx1,61$}
				\end{alakohdat}
				\end{vastaus}
\end{tehtava}

\begin{tehtava} 
(S2013/*15) Suora ympyrälierö on pallon sisällä niin, että sen molempien pohjien reunat sivuavat pallon pintaa. Pallon pinta-alan suhdetta lieriön koko pinta-alaan merkitään symbolilla $t$. Lieriön koko pinta-alalla tarkoitetaan sen vaipan ja pohjien yhteenlaskettuja pinta-aloja.
  \begin{alakohdat}
    \alakohta{Määritä lieriön korkeuden suhde lieriön pohjan säteeseen parametrin $t$ avulla lausuttuna. (2p.)
    Millä parametrin $t$ arvoilla}
    \alakohta{tällaista lieriöitä ei ole olemassa (2p.)}
    \alakohta{on täsmälleen yksi tällainen lieriö? (3p.)}
    \alakohta{on kaksi tällaista lieriötä? (2p.)}
  \end{alakohdat}
				\begin{vastaus}
				\begin{alakohdat}
				 \alakohta{$x=t\pm\sqrt{t^2+2t-4}$}
				 \alakohta{$0<t<-1+\sqrt{5}$}
				 \alakohta{$t\geq2$}
				 \alakohta{$\sqrt{5}-1<t<2$}
				\end{alakohdat}
				\end{vastaus}
\end{tehtava}