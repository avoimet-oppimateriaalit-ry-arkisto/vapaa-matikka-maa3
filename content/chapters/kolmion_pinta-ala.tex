\section*{Pinta-alan laskeminen}

Kolmion pinta-alan laskemiseen on olemassa useampia kaavoja, joita voi käyttää käytettävissä olevien tietojen perusteella. Esimerkiksi jos kolmion korkeutta ei tiedetä, mutta tiedossa on kaksi kolmion sivua ja niiden välinen kulma voidaan käyttää  kolmion pinta-alan trigonometrista kaavaa, eikä korkeutta tarvitse erikseen määrittää.

% FIXME: perustelu suunnikkail?
\begin{luoKuva}{kolmiopintaala}
	rajaa(minX = -1, maxX = 6)
	A = (0, 0)
	B = (5, 0)
	C = (3, 3)
	P = (3, 0)
	geom.suora(A, B, "$s$")
	geom.jana(B, C)
	geom.jana(C, A)
	geom.jana(P, C, "$h$")
	geom.suorakulma(C, P, A)
\end{luoKuva}
\laatikko{
Kolmion pinta-ala on puolet kolmion sivun pituuden ja saman sivun määräämälle suoralle piirretyn korkeusjanan pituuden tulosta.
\[A = \frac12sh\]
\begin{center}
\naytaKuva{kolmiopintaala}
\end{center}
}

Kolmion pinta-alan trigonometrinen kaava:

\laatikko{
Kolmion sivut ovat a,b ja c ja niitä vastaavat kulmat ovat $\alpha$, $\beta$ ja $\gamma$. Kolmion pinta-ala on
$$A= \frac{1}{2} a b \sin \gamma.$$
}

Pinta-alan voi myös laskea niin sanotulla Heronin kaavalla: 

\laatikko{
Kolmion sivut ovat $a$,$b$ ja $c$. $p$ on puolet kolmion piiristä (ns. \termi{puolipiiri}{puolipiiri} eli 
\[p = \frac{a+b+c}{2} \]
Kolmion pinta-ala on 
$$A = \sqrt{p(p-a)(p-b)(p-c)}.$$
}


