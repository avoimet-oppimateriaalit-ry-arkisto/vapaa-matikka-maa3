\section*{Pinta-alan laskeminen}

% FIXME: perustelu suunnikkail?
\begin{luoKuva}{kolmiopintaala}
	rajaa(minX = -1, maxX = 6)
	A = (0, 0)
	B = (5, 0)
	C = (3, 3)
	P = (3, 0)
	geom.suora(A, B, "$s$")
	geom.jana(B, C)
	geom.jana(C, A)
	geom.jana(P, C, "$h$")
	geom.suorakulma(C, P, A)
\end{luoKuva}
\laatikko{
Kolmion pinta-ala on puolet kolmion sivun pituuden ja saman sivun määräämälle suoralle piirretyn korkeusjanan pituuden tulosta.
\[A = \frac12sh\]
\begin{center}
\naytaKuva{kolmiopintaala}
\end{center}
}

\laatikko{
Oletetaan, että kolmion sivut ovat a,b ja c ja niitä vastaavat kulmat ovat $\alpha$, $\beta$ ja $\gamma$. Kolmion pinta-ala on
$$A= \frac{1}{2} a b \sin \gamma.$$
}
