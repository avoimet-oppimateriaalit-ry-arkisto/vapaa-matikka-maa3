\section*{Kolmion kulmien summa}

Kolmion kulmien summa on aina sama, mikä on erittäin hyödyllistä. Minkä tahansa kolmion kulmien summa on $ 180^{\circ}$ , eli yhtä suuri kuin oikokulma.

Tarkastellaan kolmiota ABC:

Piirretään kärjen C kautta kulkeva janan AB kanssa yhden suuntainen suora L

... tähän kaunis kuva ...

Suorien L ja AC välinen kulma on kulman $ {\alpha}$ vastinkulma, ja näin ollen yhtä suuri.
Vastaavasti suorien L ja CB välinen kulma on yhtäsuuri kuin kulma $ {\beta}$.

Nyt, pisteen C ja suoran L rajaaman kulman suuruus on $ {\alpha + \beta + \gamma = 180^\circ} $, eli kolmion kulmien summa on oikokulma ${\square}$
 



 