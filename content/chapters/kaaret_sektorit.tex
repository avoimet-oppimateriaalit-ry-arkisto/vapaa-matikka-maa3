\section*{Kaaret ja sektorit}

\begin{luoKuva}{keskuskulmakaari}
	skaalaa(2)
	O = geom.piste(0, 0)
	w = geom.ympyra(O, 1)
	f = lambda x: (cos(pi * x / 180), sin(pi * x / 180))
	a = -45
	c = 45
	A = f(a)
	C = f(c)
	geom.jana(A, O)
	geom.jana(O, C)
	geom.kulma(A, O, C)
	with paksuus(3):
		geom.kaari(O, 1, a, c)
\end{luoKuva}
\begin{luoKuva}{keskuskulmasektori}
	skaalaa(2)
	O = geom.piste(0, 0)
	w = geom.ympyra(O, 1)
	f = lambda x: (cos(pi * x / 180), sin(pi * x / 180))
	a = -45
	c = 45
	A = f(a)
	C = f(c)
	geom.kulma(A, O, C)
	with paksuus(3):
		geom.jana(A, O)
		geom.jana(O, C)
		geom.kaari(O, 1, a, c)
\end{luoKuva}
\begin{luoKuva}{segmentti}
	skaalaa(2)
	O = geom.piste(0, 0)
	w = geom.ympyra(O, 1)
	f = lambda x: (cos(pi * x / 180), sin(pi * x / 180))
	a = -45
	c = 45
	A = f(a)
	C = f(c)
	with paksuus(3):
		geom.jana(A, C)
		geom.kaari(O, 1, a, c)
\end{luoKuva}
\laatikko{
Kulmaa, jonka kärki on ympyrän keskipisteessä ja kyljet kulkevat kaaren päätepisteiden kautta, kutsutaan \termi{keskuskulma}{keskuskulmaksi}. Keskuskulma rajaa ympyrän kehältä \termi{kaari}{kaaren}.

\begin{center}
\naytaKuva{keskuskulmakaari}
\end{center}

Keskuskulman kylkien ja keskuskulmaa vastaava kaaren rajaamaa aluetta kutsutaan \termi{sektori}{sektoriksi}. Kaaren ja sen päätepisteiden välisen jänteen rajaamaa aluetta kutsutaan \termi{segmentti}{segmentiksi}.

% FIXME: väritä!
\begin{center}
\begin{minipage}{4.5cm}
\naytaKuva{keskuskulmasektori}
\end{minipage}
\begin{minipage}{4.5cm}
\naytaKuva{segmentti}
\end{minipage}
\end{center}
}

