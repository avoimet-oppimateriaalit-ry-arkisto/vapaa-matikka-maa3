\section*{Peruskäsitteitä}

Monet seuraavista käsitteistä on esitelty jo peruskoulussa, mutta palautetaan ne lyhyesti
mieleen.

Tasogeometrian peruskäsitteet ovat \termi{piste}{piste} ja \termi{suora}{suora}, joita ei
määritellä sen tarkemmin. (selitä, että vastaavat intuitiota)

Pisteitä merkitään useimmiten isoilla kirjaimilla, esim. $A$, $B$, $P$. Suoria merkitään useimmiten pienillä kirjamilla, esim. $l$, $s$. Suora voidaan nimetä myös kahden pisteensä mukaan, esimerkiksi suoraa, joka sisältää pisteet $A$ ja $B$ voidaan kutsua suoraksi $AB$.

Suoralla oleva piste jakaa suoran kahdeksi \termi{puolisuora}{puolisuoraksi}. Huomaa, että tämä puolisuoran
alkupiste kuuluu kumpaankin puolisuoraan. Kaksi suoralla olevaa pistettä erottavat suorasta
\termi{jana}{janan}, ja nämä kaksi pistettä ovat janan päätepisteet.

Pisteiden $A$ ja $B$ välistä janaa kutsutaan janaksi $AB$. Puolisuorat nimetään päätepisteen ja toisen puolisuoran pisteen mukaan. Siis puolisuora $AB$ on se puolisuora, jonka päätepiste on $A$ ja johon kuuluu toinen piste $B$.

\begin{esimerkki}
Suora $s$ sisältää pisteet $A$, $B$ ja $P$ mutta ei pistettä $C$. Suora $l$ sisältää pisteet $B$ ja $C$ mutta ei pisteitä $A$ ja $P$. Suoraa $s$ voidaan kutsua myös suoraksi $AB$, $AP$ tai $BP$ ja suoraa $l$ suoraksi $BC$.

Piste $B$ jakaa suoran $s$ kahteen puolisuoraan, toinen sisältää pisteen $A$ ja toinen pisteen $P$. Näitä puolisuoria voidaan kutsua esimerkiksi puolisuoriksi $BA$ ja $BP$.

\begin{center}
\begin{kuva}
	rajaa(minX = -4, maxX = 4, minY = -2, maxY = 6)
	A = geom.piste(-3, 0, "$A$", suunta = -60)
	B = geom.piste(1, 2, "$B$", suunta = 80)
	C = geom.piste(-2, 5, "$C$", suunta = 50)
	P = geom.piste(3, 3, "$P$", suunta = -60)
	s = geom.suora(A, B, "$s$")
	l = geom.suora(B, C, "$l$")
\end{kuva}
\end{center}
\end{esimerkki}

Kaksi puolisuoraa, joilla on yhteinen alkupiste, jakavat tason kahteen osaan ja rajaavat siten kaksi \termi{kulma}{kulmaa}. Yhteistä alkupistettä
kutsutaan kulman kärjeksi ja puolisuoria kulman kyljiksi. Kulman sisälle jäävää tasoaluetta
kutsutaan kulman aukeamaksi.

\begin{center}
\begin{kuva}
	skaalaa(0.5)
	A = (4, -3)
	B = geom.piste(0, 0, r"k\"{a}rki", suunta = 180)
	C = (4, 3)
	geom.jana(B, A, "oikea kylki", kohta = 0.8)
	geom.jana(B, C, "vasen kylki", puoli = False, kohta = 0.8)
	geom.kulma(A, B, C, "aukeama")
\end{kuva}
\end{center}

Kulman suuruutta on tähän mennessä mitattu \termi{aste}{asteilla} (merkitään symbolilla $^{\circ}$). Jos kulman kyljet muodostavat suoran,
kulmaa sanotaan oikokulmaksi, ja sen suuruus on $180^{\circ}$.
Jos kulman kyljet ovat sama puolisuora, niin kulma on täysikulma ($360^{\circ}$), jos aukeamaksi lasketaan
koko taso. Jos taas kulman aukeama on pelkästään tämä puolisuora, kulmaa sanotaan
nollakulmaksi ($0^{\circ}$).

Toisinaan asteet jaetaan vielä minuutteihin ja sekunteihin. Yksi aste on $60$ minuuttia ja
yksi minuutti on $60$ sekuntia ($1^{\circ} = 60^{\prime}$ ja $1^{\prime} = 
60^{\prime \prime}$).
Käytössä on myös muitakin tapoja mitata kulmia. Erityisen tärkeä asema on radiaaneilla, jotka
ovat monissa tilanteissa luonteva tapa mitata kulmia vastaavien kaarten pituuksilla. Radiaaneihin
tutustutaan tarkemmin kurssilla Trigonometriset funktiot ja lukujonot.

\laatikko{
Huomautus!
Kannattaa aina tarkistaa, mitä kulmayksikköä laskimesi käyttää. Käyttöohjeesta käy ilmi,
miten voit halutessasi vaihtaa yksikön asteiksi tai radiaaneiksi.
}

Jos kulma on
suuruudeltaan oikokulman ja täysikulman välissä, kulma on kupera. Jos taas kulma on pienempi
kuin oikokulma mutta suurempi kuin nollakulma, se on kovera kulma.

Koverat kulmat voidaan jakaa kolmeen eri tyyppiin. Kulmaa, joka on suuruudeltaan puolet
oikokulmasta, kutsutaan suoraksi kulmaksi. Suoraa kulmaa pienemmät kulmat ovat teräviä
kulmia ja suoraa kulmaa suuremmat kulmat ovat tylppiä kulmia.

\begin{minipage}{3cm}
\begin{kuva}
skaalaa(1.5); rajaa(minX = -1, minY = -0.5, maxX = 1, maxY = 1); varaaRajaus()
A = (sin(1), cos(1))
B = (0, 0); C = (-A[0], A[1]); geom.jana(B, A); geom.jana(B, C); geom.kulma(A, B, C, r"$\alpha$")
\end{kuva}
\centering kovera kulma \\$0^\circ < \alpha < 180^\circ$
\end{minipage}
\begin{minipage}{3cm}
\begin{kuva}
skaalaa(1.5); rajaa(minX = -1, minY = -0.5, maxX = 1, maxY = 1); varaaRajaus()
A = (sin(pi-1), cos(pi-1))
B = (0, 0); C = (-A[0], A[1]); geom.jana(B, A); geom.jana(B, C); geom.kulma(A, B, C, r"$\alpha$")
\end{kuva}
\centering kupera kulma \\$180^\circ < \alpha < 360^\circ$
\end{minipage}
\begin{minipage}{3cm}
\begin{kuva}
skaalaa(1.5); rajaa(minX = -1, minY = -0.5, maxX = 1, maxY = 1); varaaRajaus()
A = (sin(0.5), cos(0.5))
B = (0, 0); C = (-A[0], A[1]); geom.jana(B, A); geom.jana(B, C); geom.kulma(A, B, C, r"$\alpha$")
\end{kuva}
\centering terävä kulma \\$0^\circ < \alpha < 90^\circ$
\end{minipage}
\begin{minipage}{3cm}
\begin{kuva}
skaalaa(1.5); rajaa(minX = -1, minY = -0.5, maxX = 1, maxY = 1); varaaRajaus()
A = (sin(1), cos(1))
B = (0, 0); C = (-A[0], A[1]); geom.jana(B, A); geom.jana(B, C); geom.kulma(A, B, C, r"$\alpha$")
\end{kuva}
\centering tylppä kulma \\$90^\circ < \alpha < 180^\circ$
\end{minipage}

\begin{minipage}{3cm}
\begin{kuva}
skaalaa(1.5); rajaa(minX = -1, minY = -0.5, maxX = 1, maxY = 1); varaaRajaus()
A = (sin(pi/4), cos(pi/4))
B = (0, 0); C = (-A[0], A[1]); geom.jana(B, A); geom.jana(B, C); geom.suorakulma(A, B)
\end{kuva}
\centering suora kulma \\$\alpha = 90^\circ$
\end{minipage}
\begin{minipage}{3cm}
\begin{kuva}
skaalaa(1.5); rajaa(minX = -1, minY = -0.5, maxX = 1, maxY = 1); varaaRajaus()
A = (sin(pi/2), cos(pi/2))
B = (0, 0); C = (-A[0], A[1]); geom.jana(B, A); geom.jana(B, C); geom.kulma(A, B, C, r"$\alpha$")
\end{kuva}
\centering oikokulma \\$\alpha = 180^\circ$
\end{minipage}
\begin{minipage}{3cm}
\begin{kuva}
skaalaa(1.5); rajaa(minX = -1, minY = -0.5, maxX = 1, maxY = 1); varaaRajaus()
geom.jana((0, 0), (0, 1))
\end{kuva}
\centering nollakulma \\$\alpha = 0^\circ$
\end{minipage}
\begin{minipage}{3cm}
\begin{kuva}
skaalaa(1.5); rajaa(minX = -1, minY = -1, maxX = 1, maxY = 0.5); varaaRajaus()
A = (sin(pi - 0.001), cos(pi - 0.001))
B = (0, 0); C = (-A[0], A[1]); geom.jana(B, A); geom.jana(B, C); geom.kulma(A, B, C, r"$\alpha$")
\end{kuva}
\centering täysikulma \\$\alpha = 360^\circ$
\end{minipage}

Mikäli kulman oikealla kyljellä oleva piste, kärkipiste ja vasemmalla kyljellä oleva piste on nimetty kirjaimella, voidaan kulmaa merkitä kirjoittamalla kirjaimet peräkkäin. Esimerkiksi kulma $\angle ABC$ merkitsee sellaista kulmaa, jonka kärkipiste on $B$ ja $A$ sijaitsee sen oikealla kyljellä ja $C$ vasemmalla. Sopimus, että oikea kylki merkitään ensin ei ole aina käytössä, siis joskus merkinnällä $\angle ABC$ voidaan tarkoittaa kulmaa $\angle CBA$.

\begin{esimerkki}
\begin{center}
\begin{kuva}
	A = geom.piste(1, -1, "$A$")
	B = geom.piste(0, 0, "$B$", suunta = 180)
	C = geom.piste(1, 1, "$C$")
	geom.jana(B, A)
	geom.jana(B, C)
	geom.kulma(A, B, C, r"$\alpha$")
	
	D = geom.piste(3, -1, "$D$", suunta = 180)
	E = geom.piste(4, 0, "$E$", suunta = 180)
	F = geom.piste(3, 1, "$F$", suunta = 180)
	geom.jana(E, D)
	geom.jana(E, F)
	geom.kulma(D, E, F, r"$\beta$")
\end{kuva}
\end{center}
\[
\begin{array}{cc}
\displaystyle \angle ABC = \alpha, &
\displaystyle \angle DEF = \beta.
\end{array}
\]
\end{esimerkki}


