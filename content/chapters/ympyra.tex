\section*{Ympyrä}

\begin{luoKuva}{ympyraosat}
	skaalaa(2)
	keskipiste = geom.piste(0, 0, "keskipiste")
	ympyra = geom.ympyra(keskipiste, 1, r"ympyr\"{a}", 30)
	kehapiste = geom.leikkauspiste(
		ympyra, geom.suoraSuuntaan(keskipiste, -1, 2, piirra = False), piirra = False
	)
	geom.jana(kehapiste, keskipiste, r"s\"{a}de")
\end{luoKuva}
\laatikko{
\termi{ympyrä}{Ympyrä} on sellaisten pisteiden joukko, jotka ovat kiinteällä etäisyydellä tietystä pisteestä. Tätä kiinteää etäisyyttä kutsutaan ympyrän \termi{säde}{säteeksi} ja kyseistä pistettä ympyrän \termi{keskipiste}{keskipisteeksi}.
}
\begin{center}
\naytaKuva{ympyraosat}
\end{center}

\begin{esimerkki}
Kuu kiertää Maata likimäärin ympyränmuotoisella radalla, jonka keskipiste on Maa ja säde on Kuun ja Maan välinen etäisyys.
\end{esimerkki}

(KUVA TÄHÄN)

\begin{luoKuva}{ympyrajanne}
	skaalaa(2)
	O = geom.piste(0, 0)
	w = geom.ympyra(O, 1)
	s = geom.suoraSuuntaan(O, 1, 0, piirra = False)
	l = geom.suoraSuuntaan((0.2, 0.6), 1, -0.2, piirra = False)
	A = geom.leikkauspiste(w, s, valinta = False, piirra = False)
	B = geom.leikkauspiste(w, s, valinta = True, piirra = False)
	C = geom.leikkauspiste(w, l, valinta = False, piirra = False)
	D = geom.leikkauspiste(w, l, valinta = True, piirra = False)
	geom.jana(A, B, "halkaisija")
	geom.jana(C, D, r"j\"{a}nne")
\end{luoKuva}
\laatikko{
Ympyrän \termi{jänne}{jänne} on jana kahden ympyrän kehäpisteen välillä. Mikäli jänne kulkee ympyrän keskipisteen kautta, se on ympyrän \termi{halkaisija}{halkaisija}. Halkaisijan pituus on kaksi kertaa ympyrän säde.
}
\begin{center}
\naytaKuva{ympyrajanne}
\end{center}


