\section*{Ympyrän tangetti}

... jtn yleistä löpinää? ...

\laatikko{
	\termi \textbf{Ympyrän tangentti} on suora joka leikkaa ympyrän tasan yhdessä pisteessä, toisin sanoen se sivuaa ympyrää.
	
	Ympyrän tangentti on aina kohtisuorassa sivuamispisteeseen piirrettyä sädettä vastaan.
	
	Jonkin ympyrän ulkopuolisen pisteen kautta kulkee kaksi ympyrän tangenttia.	 
	}

%\begin{kuva}
%rajaa(minX = -1, maxX = 5)
%rajaa(minY = -1, maxY = 5)

%O = geom.piste(1,0)
%Y = geom.ympyra(O,1,piirra = True)
%A = geom.ympyranKehapiste(Y,90,"", suunta = 50)
%OA = geom.jana(A,O)
%P = geom.piste(1,2) 

%\end{kuva}

... tähän kuvitusta ... 

\begin{esimerkki}

tähän hieno esimerkki tangentin käytöstä

\end{esimerkki}

%\paragraph*{Opi perusteet}

%\paragraph*{Hallitse kokonaisuus}

%\paragraph*{Sekalaisia tehtäviä}


