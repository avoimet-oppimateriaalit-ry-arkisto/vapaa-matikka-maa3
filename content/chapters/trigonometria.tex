
\section*{Trigonometria}

Tarkastellaan kahta suorakulmaista kolmiota, joilla on yhtä suuri terävä kulma $\alpha$.
Silloin kolmiot ovat keskenään yhdenmuotoisia, sillä niillä on kaksi yhtä suurta vastinkulmaa.
Yhdenmuotoisuuden nojalla sivujen keskinäiset suhteet ovat siis täsmälleen samoja kaikilla
suorakulmaisilla kolmioilla, joilla on samansuuruinen terävä kulma. Näille suhteet on
nimetty seuraavasti.

\laatikko{
\termi{sini}{Sini}: $\sin \alpha = \frac{\textrm{kulman $\alpha$ vastaisen kateetin pituus}}
{\textrm{hypotenuusan pituus}}$ \\
\termi{kosini}{Kosini}: $\cos \alpha = \frac{\textrm{kulman $\alpha$ viereisen kateetin pituus}}
{\textrm{hypotenuusan pituus}}$ \\
\termi{tangentti}{Tangentti}: $\tan \alpha = \frac{\textrm{kulman $\alpha$ vastaisen kateetin pituus}}
{\textrm{kulman $\alpha$ viereisen kateetin pituus}}$
}

\begin{esimerkki}
\begin{center}
\begin{kuva}
	A = (4, 3)
	B = (0, 0)
	C = (4, 0)
	geom.jana(A, B, "$c$")
	geom.jana(B, C, "$a$")
	geom.jana(C, A, "$b$")
	geom.suorakulma(A, C, B)
\end{kuva}
\end{center}
Kuvan kolmiossa pätee
\[
\begin{array}{ccc}
\displaystyle\sin \alpha = \frac{b}{c}, &
\displaystyle\cos \alpha = \frac{a}{c}, &
\displaystyle\tan \alpha = \frac{b}{a}.
\end{array}
\]
\end{esimerkki}

...yksikköympyrä, tylpän kulman siniä ja kosinia tarvitaan...