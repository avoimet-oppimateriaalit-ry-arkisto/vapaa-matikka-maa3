\section*{Kehäkulmalause}

Ympyrän \termi{kehäkulma}{kehäkulmalla} tarkoitetaan kulmaa, jonka kärki on ympyrän kehällä ja kyljet ympyrän jänteitä. Kehäkulma erottaa ympyrän kehältä kaaren.
\begin{center}
\begin{kuva}
	skaalaa(2)
	O = geom.piste(0, 0, piirra = False)
	w = geom.ympyra(O, 1)
	f = lambda x: (cos(pi * x / 180), sin(pi * x / 180))
	a = -60
	c = 30
	A = f(a)
	B = f(135)
	C = f(c)
	geom.jana(A, B)
	geom.jana(B, C)
	geom.kulma(A, B, C)
	with paksuus(3):
		geom.kaari(O, 1, a, c)
\end{kuva}
\end{center}

\begin{luoKuva}{kehakulmalause}
	skaalaa(2)
	O = geom.piste(0, 0)
	w = geom.ympyra(O, 1)
	f = lambda x: (cos(pi * x / 180), sin(pi * x / 180))
	a = -70
	b = 20
	A = f(a)
	B = f(b)
	X = f(140)
	
	geom.jana(X, A)
	geom.jana(X, B)
	geom.jana(O, A)
	geom.jana(O, B)
	
	geom.kulma(A, X, B, r"$\alpha$")
	geom.kulma(A, O, B, r"$2\alpha$")
\end{luoKuva}

\laatikko{
\termi{kehäkulmalause}{Kehäkulmalause}

Kaarta vastaava kehäkulman suuruus on puolet samaa kaarta vastaavan keskuskulman suuruudesta. Erityisesti kaikki samaa kaarta vastaavat kehäkulmat ovat yhtä suuria.

\begin{center}
\naytaKuva{kehakulmalause}
\end{center}
}
