\section*{Kehäkulmalause}

\begin{luoKuva}{kehakulma}
skaalaa(2)
O = geom.piste(0, 0, piirra = False)
w = geom.ympyra(O, 1)
f = lambda x: (cos(pi * x / 180), sin(pi * x / 180))
A = f(-60)
B = f(135)
C = f(30)
geom.jana(A, B)
geom.jana(B, C)
geom.kulma(A, B, C)
\end{luoKuva}
\laatikko{
Ympyrän \termi{kehäkulma}{kehäkulmalla} tarkoitetaan kulmaa, jonka kärki on ympyrän kehällä ja kyljet ympyrän jänteitä. Kehäkulma erottaa ympyrän kehältä kaaren.
\begin{center}
\naytaKuva{kehakulma}
\end{center}
}
