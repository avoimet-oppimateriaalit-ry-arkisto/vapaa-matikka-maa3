\section*{Peruskäsitteitä}

Monet seuraavista käsitteistä on esitelty jo peruskoulussa, mutta palautetaan ne lyhyesti
mieleen.

Tasogeometrian peruskäsitteet ovat \termi{piste}{piste} ja \termi{suora}{suora}, joita ei
määritellä sen tarkemmin. (selitä, että vastaavat intuitiota)

Suoralla oleva piste jakaa suoran kahdeksi puolisuoraksi. Huomaa, että tämä puolisuoran
alkupiste kuuluu kumpaankin puolisuoraan. Kaksi suoralla olevaa pistettä erottavat suorasta
janan, ja nämä kaksi pistettä ovat janan päätepisteet.

Kaksi puolisuoraa, joilla on yhteinen alkupiste, rajaavat kulman. Yhteistä alkupistettä
kutsutaan kulman kärjeksi ja puolisuoria kulman kyljiksi. 