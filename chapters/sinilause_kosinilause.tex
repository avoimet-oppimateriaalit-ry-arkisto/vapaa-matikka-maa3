\section*{Sinilause ja kosinilause}

\begin{luoKuva}{sinilause}
A = geom.piste(0, 0, piirra = False)
B = geom.piste(5, 2, piirra = False)
C = geom.piste(2, 4, piirra = False)
geom.jana(A, B, "$c$", kohta = 0.4)
geom.jana(B, C, "$a$")
geom.jana(C, A, "$b$")
geom.kulma(B, A, C, r"$\alpha$")
geom.kulma(C, B, A, r"$\beta$")
geom.kulma(A, C, B, r"$\gamma$")

w = geom.ymparipiirrettyYmpyra(A, B, C)
O = geom.ympyranKeskipiste(w)
R = geom.leikkauspiste(w, geom.suora(O, (6, 0), piirra = False), piirra = False)
geom.jana(O, R, "$r$", kohta = 0.7)
\end{luoKuva}

\laatikko{
Oletetaan, että kolmion sivut ovat a,b ja c ja niitä vastaavat kulmat ovat $\alpha$, $\beta$ ja $\gamma$. Sinilauseen mukaan
$$\frac{a}{\sin \alpha} = \frac{b}{\sin \beta} = \frac{c}{\sin \gamma} = 2r,$$
missä r on kolmion ympäri piirretyn ympyrän säde.

\begin{center}
\naytaKuva{sinilause}
\end{center}
}

Sinilauseesta voidaan johtaa kaava kolmion pinta-alalle. Tämän kaavan avulla voidaan laskea kolmion pinta-ala, kun tiedetään kaksi sivua ja niiden välinen kulma.

\laatikko{
Oletetaan, että kolmion sivut ovat a,b ja c ja niitä vastaavat kulmat ovat $\alpha$, $\beta$ ja $\gamma$. Kolmion pinta-ala on
$$A= \frac{1}{2} \cdot a \cdot b \cdot \sin \gamma.$$
}

\begin{esimerkki}
Kolmion kahden sivun pituudet ovat 5cm ja 7cm, ja näistä pitempää sivua vastaava kulma on $60^\circ$. Laske kolmion kolmannen sivun pituus, loppujen kulmien suuruus sekä kolmion ala.

\textbf{Ratkaisu.}

Lasketaan ensin 5cm sivua vastaavan kulman suuruus. Tehdään verrantoyhtälö:
$$\frac{7}{\sin 60} = \frac{5}{\sin \beta} | \cdot \sin 60, \cdot \sin \beta$$
$$7 \cdot \sin \beta = 5 \cdot \sin 60 | / 7$$
$$\sin \beta = \frac{5 \cdot \sin 60}{7}$$
$$\beta \approx 38,2^\circ$$
Siis 5cm mittaista sivua vastaava kulma on noin $38,2^\circ$. Kolmannen kulman suuruus on siis $180^\circ - 60^\circ -38,2^\circ = 81,8^\circ4$.
Lasketaan seuraavaksi kolmannen sivun pituus. Tehdään verrantoyhtälö:
$$\frac{7}{\sin 60} = \frac{c}{\sin 81,8}| \cdot \sin 60, \cdot \sin 81,8$$
$$7 \cdot \sin 81,8 = c \cdot \sin 60 | / \sin 60$$
$$\frac{7 \cdot \sin 81,8}{\sin 60} = c$$
$$c \approx 8$$
Siis kolmannen sivun pituus on noin 8cm.

Lasketaan lopuksi kolmion pinta-ala:
$$A = \frac{1}{2} \cdot 5 \cdot 7 \cdot \sin 81,8 \approx 17.$$
Siis kolmion pinta-ala on noin 17 cm².
\end{esimerkki}