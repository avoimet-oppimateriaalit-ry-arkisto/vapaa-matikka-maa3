\section*{Sinilause ja kosinilause}

\begin{luoKuva}{sinilause}
A = geom.piste(0, 0, piirra = False)
B = geom.piste(5, 2, piirra = False)
C = geom.piste(2, 4, piirra = False)
geom.jana(A, B, "$c$", kohta = 0.35)
geom.jana(B, C, "$a$")
geom.jana(C, A, "$b$")
geom.kulma(B, A, C, r"$\alpha$")
geom.kulma(C, B, A, r"$\beta$")
geom.kulma(A, C, B, r"$\gamma$")

w = geom.ymparipiirrettyYmpyra(A, B, C)
O = geom.ympyranKeskipiste(w)
R = geom.ympyranKehapiste(w, -20, piirra = False)
geom.jana(O, R, "$R$", kohta = 0.6)
\end{luoKuva}

\laatikko{
Oletetaan, että kolmion sivut ovat $a$, $b$ ja $c$, ja niitä vastaavat kulmat ovat $\alpha$, $\beta$ ja $\gamma$. \termi{sinilause}{Sinilauseen} mukaan
$$\frac{a}{\sin \alpha} = \frac{b}{\sin \beta} = \frac{c}{\sin \gamma} = 2R,$$
missä $R$ on kolmion ympäri piirretyn ympyrän säde.

\begin{center}
\naytaKuva{sinilause}
\end{center}
}

\textbf{Sinilauseen todistus.} Tarkastellaan sivulle $c$ piirrettyä korkeusjanaa.
Sen pituus $h$ on suorakulmaisen kolmion kateettina, joten sen voi ilmoittaa muodossa
\[
h = a \sin \beta
\]
tai käyttämällä toista suorakulmaista kolmiota, muodossa
\[
h = b \sin \alpha .
\]

Tällöin on voimassa
\[
a \sin \beta = b \sin \alpha
\]
eli
\[
\frac{a}{\sin \alpha} = \frac{b}{\sin \beta}.
\]

Samalla tavoin päätellään, että
\[
\frac{b}{\sin \beta} = \frac{c}{\sin \gamma}.
\]

Viimeisen yhtäsuuruuden voimassaolo perustellaan kehäkulmalauseen yhteydessä.

Sinilauseesta voidaan johtaa kaava kolmion pinta-alalle. Tämän kaavan avulla voidaan laskea kolmion pinta-ala, kun tiedetään kaksi sivua ja niiden välinen kulma.

\laatikko{
Oletetaan, että kolmion sivut ovat a,b ja c ja niitä vastaavat kulmat ovat $\alpha$, $\beta$ ja $\gamma$. Kolmion pinta-ala on
$$A= \frac{1}{2} a b \sin \gamma.$$
}

\begin{esimerkki}
Kolmion kahden sivun pituudet ovat 5cm ja 7cm, ja näistä pitempää sivua vastaava kulma on $60^\circ$. Laske kolmion kolmannen sivun pituus, loppujen kulmien suuruus sekä kolmion ala.

\textbf{Ratkaisu.}

Lasketaan ensin 5cm sivua vastaavan kulman suuruus. Tehdään verrantoyhtälö:
$$\frac{7}{\sin 60} = \frac{5}{\sin \beta} | \cdot \sin 60, \cdot \sin \beta$$
$$7 \cdot \sin \beta = 5 \cdot \sin 60 | / 7$$
$$\sin \beta = \frac{5 \cdot \sin 60}{7}$$
$$\beta \approx 38,2^\circ$$
Siis 5cm mittaista sivua vastaava kulma on noin $38,2^\circ$. Kolmannen kulman suuruus on siis $180^\circ - 60^\circ -38,2^\circ = 81,8^\circ4$.
Lasketaan seuraavaksi kolmannen sivun pituus. Tehdään verrantoyhtälö:
$$\frac{7}{\sin 60} = \frac{c}{\sin 81,8}| \cdot \sin 60, \cdot \sin 81,8$$
$$7 \cdot \sin 81,8 = c \cdot \sin 60 | / \sin 60$$
$$\frac{7 \cdot \sin 81,8}{\sin 60} = c$$
$$c \approx 8$$
Siis kolmannen sivun pituus on noin 8cm.

Lasketaan lopuksi kolmion pinta-ala:
$$A = \frac{1}{2} \cdot 5 \cdot 7 \cdot \sin 81,8 \approx 17.$$
Siis kolmion pinta-ala on noin 17 cm².
\end{esimerkki}

\laatikko{
Olkoot kolmion sivut $a$, $b$ ja $c$, ja niitä vastaavat kulmat $\alpha$, $\beta$ ja
$\gamma$. Tällöin pätee seuraava \termi{kosinilause}{kosinilauseena} tunnettu kaava.
\[
c^2 = a^2 + b^2 - 2 a b \cos \gamma
\]
}

\textbf{Kosinilauseen todistus}. Olkoon sivulle $a$ piirretty korkeusjana $h$, ja tämän
korkeusjanan kantapisteen etäisyys kulman $\gamma$ kärjestä $d$ ja kulman $\beta$ kärjestä $e$.

(KUVA)

Pyhtagoraan lauseen nojalla seuraavat yhtälöt ovat voimassa.

\[
h^2 + d^2 = b^2
\]
\[
h^2 + e^2 = c^2
\]

Oletetaan ensin, että kulma $\gamma$ ei ole tylppä. Tällöin $d = b \cos \gamma$.
Lisäksi joko $e = a - d$ tai $e = d - a$ riippuen siitä, onko kulma $\beta$ terävä vai
tylppä. Joka tapauksessa $e^2 = a^2 - 2ad + d^2$, ja siksi

\begin{align*}
c^2 = h^2 + e^2 \\
= b^2 - d^2 + e^2 \\
= b^2 - d^2 + a^2 - 2 a d + d^2 \\
= a^2 + b^2 - 2 a b \cos \gamma .
\end{align*}

Oletetaan sitten, että kulma $\gamma$ on tylppä. Koska $\cos (180^{\circ} - \gamma )
= - \cos \gamma$, niin $d = -b \cos \gamma$. Lisäksi $e = a + d$. Näin saadaan

\begin{align*}
c^2 = h^2 + e^2 \\
= b^2 - d^2 + e^2 \\
= b^2 - d^2 + a^2 + 2 a d + d^2 \\
= a^2 + b^2 + 2 a(- b \cos \gamma ) \\
= a^2 + b^2 - 2 a b \cos \gamma .
\end{align*}

Kaava on siis voimassa. $\square$