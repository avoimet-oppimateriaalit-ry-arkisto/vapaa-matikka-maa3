\section*{Ympyrä}

\laatikko{
\termi{ympyrä}{Ympyrä} on sellaisten pisteiden joukko, jotka ovat kiinteällä etäisyydellä tietystä pisteestä. Tätä kiinteää etäisyyttä kutsutaan ympyrän \termi{säde}{säteeksi} ja kyseistä pistettä ympyrän \termi{keskipiste}{keskipisteeksi}.
}

\begin{kuva}
keskipiste = geom.piste(0, 0, "keskipiste")
kehapiste = geom.piste(-1, 2)
ympyra = geom.ympyra(keskipiste, geom.etaisyys(keskipiste, kehapiste), r"ympyr\"{a}", 30)
geom.jana(kehapiste, keskipiste, r"s\"{a}de")
\end{kuva}

\begin{esimerkki}
Kuu kiertää Maata likimäärin ympyränmuotoisella radalla, jonka keskipiste on Maa ja säde on Kuun ja Maan välinen etäisyys.
\end{esimerkki}

(KUVA TÄHÄN)

\laatikko{

}