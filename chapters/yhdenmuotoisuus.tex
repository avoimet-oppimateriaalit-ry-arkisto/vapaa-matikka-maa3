\section*{Yhdenmuotoisuus}

Jos jokin tasokuvio on saatu toisesta tasokuviosta siirroilla, kierroilla tai
peilauksilla, niin näitä kuvioita sanotaan yhteneviksi. Koska pituudet ja kulmat säilyvät
näissä operaatioissa, niin näillä kuvioilla toisiaan vastaavat kulmat ja janat
ovat yhtä suuret.

Jos yhtenevää kuviota suurennetaan tai pienennetään, niin kulmat pysyvät edelleen yhtäsuurina.
Vaikka pituudet eivät pysykään samoina, niin toisiaan vastaavien janojen pituuksien suhde
on sama kaikille janoille. Sanomme, että tässä tilanteessa kuviot ovat yhdenmuotoiset.

\laatikko{
	Kaksi kuviota ovat keskenään yhdenmuotoiset, jos niiden pisteille löytyy vastaavuus, jolle
	
	1) Kaikki vastinkulmat ovat keskenään yhtä suuret.
	
	2) Kaikkien vastinjanojen pituuksien suhde on vakio.
}

Aina yhdenmuotoisuuden tarkistaminen ei onnistu helposti. Onneksi kolmioille on yksinkertaisia
sääntöjä, joilla yhdenmuotoisuuden perusteleminen on erityisen helppoa.

\laatikko{
	\termi{kolmioiden yhdenmuotoisuussäännöt}{Kolmioiden yhdenmuotoisuussäännöt}.
	
	Jos seuraavista ehdoista jokin on voimassa kolmioparille, niin kolmiot ovat keskenään
	yhdenmuotoiset.
	
	\begin{enumerate}
		\item Kolmioissa on kaksi paria keskenään yhtäsuuria vastinkulmia. (kk)
		\item Kolmioissa on kaksi paria keskenään verrannollisia sivuja, joiden välinen
		kulma on yhtä suuri molemmissa kolmioissa (sks)
		\item Kolmioissa on kolme paria keskenään verrannollisia sivuja. (sss)
	\end{enumerate}
}

Huomaa, että säännössä (sks) yhtäsuuren vastinkulman on oltava vastinsivujen välissä. On
nimittäin mahdollista, että kahdella kolmiolla on kaksi keskenään verrannolista sivua
ja yhtäsuuri kulma toisen sivun vieressä, mutta kolmiot eivät kuitenkaan ole yhdenmuotoiset.

(KUVA)



